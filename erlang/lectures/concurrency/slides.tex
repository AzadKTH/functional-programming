\input{../include/preamble}

\title[ID1019 Concurrency]{Concurrency}

\author{Johan Montelius}
\institute{KTH}
\date{\semester}

\begin{document}

\begin{frame}
\titlepage
\end{frame}

\begin{frame}[fragile]{concurrency}

{\em What do we mean by concurrency?}


\end{frame}


\begin{frame}{why?}

{\em Why concurrency, it's hard enough to handle one thread of execution?} 

\end{frame}

\begin{frame}{actor model}

An actor:
\begin{itemize}
\item state: keeps a private state that can only be changed by the actor
\pause
\item receive: has one channel of incoming {\em messages}
\pause
\item execute: given a state and a received message, the actor can
\begin{itemize}
\pause
\item send: send a number of messages to other actors
\pause
\item spawn: create a number of new actors
\pause
\item transform: modify its state an continue, or terminate
\end{itemize}
\end{itemize}

\end{frame}


\begin{frame}{ordering of messages}

{\em Is the set of messages to an actor ordered?}

\pause\vspace{20pt}
{\em In what order should the messages be handled?}

\pause\vspace{20pt}
{\em The evaluation of a function is deterministic, how about the execution of an actor?}


\end{frame}

\begin{frame}{naming of actors}

{\em How can an actor direct a message to a specific actor?}

\pause\vspace{20pt}
{\em Do we have a global naming scheme?}

\pause\vspace{20pt}
{\em How do we find the identifier of an actor?}
     
\end{frame}



\begin{frame}{exceptions}

{\em What should we do if we're sending a message to an actor that has terminated?}

\pause\vspace{20pt}
{\em What if we're waiting for a message that will never be sent?}

\pause\vspace{20pt}
{\em Are sent messages guaranteed to arrive?}

\end{frame}


\begin{frame}{process identifier}

We introduce one additional data structure:
\pause\vspace{20pt}

\begin{code}
 <structure> ::= <process identifier> | ...
\end{code}


\pause\vspace{20pt} 
{\em There is no term nor pattern that corresponds to an identifier.}
\end{frame}

\begin{frame}[fragile]{spawn}

  A new process is spawned by giving it a function to evaluate, \pause
  
  \vspace{10pt}\hspace{40pt}... the result is a process identifier (pid).

\pause\vspace{20pt}
\begin{verbatim}
  Pid = spawn(fun() -> ... end)
\end{verbatim}
\pause\vspace{10pt}
or
\pause\vspace{10pt}
\begin{verbatim}
  Pid = spawn(Module, Name, [Arg, ...])
\end{verbatim}
\pause\vspace{10pt}
{\em In the later case, the function must be exported from the module.}

\end{frame}

\begin{frame}[fragile]{send}

Given a process identifier, an arbitrary data structure can be sent to the process.

\pause\vspace{10pt}

\begin{verbatim}
   Pid ! Message
\end{verbatim}

\pause\vspace{20pt}
{\em The send operator {\tt !}, is often called ``bang''.}

\end{frame}

\begin{frame}{receive}

We extend expressions:

\pause\vspace{10pt}
\begin{code}
    <expression> ::=  & <receive expression> | ...
\end{code}
\pause\vspace{10pt}
\begin{code}
   <receive & expression> ::= \\
            & receive <clauses>  end
\end{code}
\pause\vspace{20pt}

{\em similar to case expressions}
\end{frame}

\begin{frame}[fragile]{example}

\begin{verbatim}
server(Sum) ->
   receive 
     {add, X} ->
           server(Sum + X);
     {sub, X} ->
           server(Sum - X)
   end.
\end{verbatim}

\end{frame}


\begin{frame}[fragile]{side effects}

In a {\em pure functional} program, the only effect of evaluating an
expressions is the returned value \pause - not any more.

\pause\vspace{10pt}
\begin{verbatim}
   Pid = spawn(fun() -> foo() end),
      :
    X = bar(Pid),
    Y = bar(Pid),
      :
\end{verbatim}

\pause\vspace{10pt}

\begin{columns}
  \begin{column}{0.5\linewidth}
\begin{verbatim}
bar(Pid) ->
    Pid ! {hello, 42}.
\end{verbatim}
  \end{column}
  \pause
  \begin{column}{0.5\linewidth}
\begin{verbatim}
bar(_) ->
   receive 
     Msg -> {hello, Msg}
   end.
\end{verbatim}
 \end{column}
\end{columns}  

\end{frame}


\begin{frame}[fragile]{side effects}

In a {\em pure functional} program, the order of execution is not important (modulo infinit and failing computations)
expressions is the returned value \pause - not any more.

\pause\vspace{10pt}
\begin{verbatim}
bar(X,Y) ->
    Z = this(X,Y),
    Q = that(X),
    {Z,Q}.
\end{verbatim}

\pause\vspace{10pt}

\begin{verbatim}
bar(Pid) ->
    Pid ! {hello, 42},
    Pid ! {hello, 17}.
\end{verbatim}

\end{frame}

 
\begin{frame}[fragile]{Who am I?}

One more built-in function:

\pause\vspace{10pt}

\begin{verbatim}
    MyPid = self()
\end{verbatim}

\end{frame}

\begin{frame}{few extensions}

Few extensions to the functional subset:
\pause\vspace{10pt}
\begin{itemize}
\pause\item pid: a process identifier as a data structure
\pause\item spawn: creating a process, returning a pid
\pause\item send: sending of messages to pids
\pause\item receive: selective receive of messages
\pause\item self: the process identifier of the current process
\end{itemize}

\pause\vspace{10pt}
{\em All constructs can, apart from the receive statement, almost be given a functional interpretation.}

\end{frame}

\begin{frame}[fragile]{order of messages}

Message passing is: unreliable FIFO.

\pause\vspace{10pt}
\begin{verbatim}
           :
        Pid ! {this, is, message, 1},
        Pid ! {this, is, message, 2},
        Pid ! {this, is, message, 3},
           :
\end{verbatim}

\pause What could be the result at the receiving end?

\pause\vspace{10pt}
{\em How many messages are lost in reality?} 

\end{frame}


\begin{frame}[fragile]{selective receive}

\begin{verbatim}
sum(S) ->
    receive 
        {add, X} -> sum(S + X);
        {sub, X} -> sum(S - X);
        {mul, X} -> sum(S * X)
    end.
\end{verbatim}

\pause\vspace{10pt}

Assume we spawn a process given the expression {\tt fun() -> sum(10) end}, \pause and the sequence of messages in the queue is:

\vspace{10pt}
{\tt \{sub, 4\}, \{add, 10\}, \{mul, 4\}, \{mul, 2\}, \{sub, 10\}}

\end{frame}



\begin{frame}[fragile]{selective receive}

\begin{columns}
 \begin{column}{0.5\linewidth}
\begin{verbatim}
closed(S) ->
    receive 
        {mul, X} -> closed(S * X);
        open -> open(S)
    end.
\end{verbatim}
 \end{column}
\pause
 \begin{column}{0.5\linewidth}
\begin{verbatim}
open(S) ->
    receive 
        {mul, X} -> open(S * X);
        {sub, X} -> open(S - X);
        close -> closed(S)
    end.
\end{verbatim}
 \end{column}
\end{columns}


\pause\vspace{10pt}

Assume we spawn {\tt fun() -> closed(10) end} and the sequence of messages is:

\vspace{10pt}

{\tt \{sub, 4\}, \{mul, 4\}, open, \{mul, 2\}, close, \{sub, 10\}}

\pause\vspace{10pt}

In every receive expression we start from the beginning of the queue.

\end{frame}


\begin{frame}{implicit deferral}

{\em Selective receive}: we specify which messages we are willing to accept.

{\em Implicit deferral}: messages that we do not explicitly receive remain in the message queue.

\pause\vspace{10pt}

Pros and cons - we could have chosen either.

\end{frame}


\begin{frame}{finite state automata}


\begin{tikzpicture}[>=stealth', node distance=3cm, semithick, auto]
    \node[initial, state]            (closed)                              {closed};
    \node[state]                     (open)    [above right of=closed]     {open};
    \node[accepting, state]          (final)   [below right of=closed]     {final};

    \path[->]   (closed) edge       [bend left]     node      {open}          (open)
                         edge       [loop below]    node      {\{mul, X\}}    (closed)
                         edge       [bend left]     node      {done}          (final)
                (open)   edge       [loop above]    node      {\{mul, X\}}    (open)
                         edge       [loop right]    node      {\{sub, X\}}    (open)
                         edge       [bend left]     node      {close}         (closed);
\end{tikzpicture}


\end{frame}


\begin{frame}{finite state automata}

Erlang receive statements are not a direct realization of a finite state
automata.

\pause\vspace{10pt}

Messages that arrive too early in a finite state automata would give
us a undefined state.

\pause\vspace{10pt}

The {\em implicit deferral} give us a very simple description of a
finite state automata where all messages are allowed to arrive too
early.



\end{frame}

\begin{frame}[fragile]{time to deliver}

\begin{columns}
   \begin{column}{.5\linewidth}
     \begin{block}{src/2}
\begin{verbatim}
src(Sink, Frw) ->
    Sink ! a,
    Frw ! b.
\end{verbatim}
      \end{block}
     \begin{block}{frw/2}
\begin{verbatim}
frw(Sink) -> 
   receive 
     Msg -> Sink ! Msg
   end.
\end{verbatim}
      \end{block}
    \end{column}
\pause
    \begin{column}{.5\linewidth}
\begin{tikzpicture}[>=stealth', node distance=3cm, semithick, auto]
    \node[state]                     (src)   []             {src};
    \node[state]                     (sink)  [right of=src]     {sink};
    \node[state]                     (frw)   [below right of=src]     {frw};

    \path[->]  (src)   edge        node           {a}       (sink)
                       edge        node           {b}       (frw);
    \pause
    \path[->]  (frw)   edge        node           {b}       (sink);

\end{tikzpicture}
    \end{column}

  \end{columns}


\end{frame}
 
\begin{frame}[fragile]{example account}

\begin{columns}
   \begin{column}{.5\linewidth}
     \begin{verbatim}
acc(Saldo) ->
  recieve
     {deposit, M} ->
        acc(Saldo + M);
     {withdraw, M} ->
        acc(Saldo - M)
  end.
     \end{verbatim}
   \end{column}
\pause
  \begin{column}{.5\linewidth}
     \begin{verbatim}
doit() ->
  Acc = spawn(fun()->acc(0) end),
  Acc ! {deposit, 20},
  Acc ! {withdraw, 10}.
  Acc.
     \end{verbatim}
   \end{column}
\end{columns}


\end{frame}

\begin{frame}[fragile]{example account}

\begin{columns}
   \begin{column}{.45\linewidth}
\begin{verbatim}
acc(Saldo) ->
  recieve
    {deposit, M} ->
       acc(Saldo + M);
    {withdraw, M} ->
       acc(Saldo - M);
    {request, Pid} ->
       Pid ! {saldo, Saldo},
       acc(Saldo)
  end.
\end{verbatim}
\end{column}
\pause
   \begin{column}{.45\linewidth}
\begin{verbatim}
check(Acc) ->
  Acc ! {request, self()},
  receive
    {saldo, Saldo} ->
        Saldo
  end.
\end{verbatim}
\end{column}
\end{columns}
\end{frame}

\begin{frame}{summary}

\begin{itemize}
\item asynchronous: messages are sent and eventually (hopefully) delivered
\item FIFO: message delivery is ordered
\item selective receive: the receiver decides the order of handling messages
\item implicit deferral: messages remain in the queue untill handled
\end{itemize}

\end{frame}


\end{document}



