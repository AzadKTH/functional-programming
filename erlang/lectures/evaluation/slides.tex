\input{../include/preamble}

\title[ID1019 Evaluation]{Evaluation}


\author{Johan Montelius}
\institute{KTH}
\date{\semester}

\begin{document}

\begin{frame}
\titlepage
\end{frame}

\begin{frame}{semantics}

We will define a small subset of the Erlang language and describe the
{\em operational semantics}.


\pause \vspace{40pt}Warning - this is not exactly how Erlang works ... but it could have been.

\end{frame}


\begin{frame}{expressions}

\begin{grammar}
<atom> ::= a | b | c | \ldots

<variable> ::= X | Y | Z | \ldots

<literal> ::= <atom>

<expression> ::= <literal> | <variable> |  '\{' <expression> ',' <expression> '\}'
\end{grammar}


\pause \vspace{20pt} Examples: {\tt \{a,b\}} , {\tt \{X,Y\}} , {\tt \{a, \{b, Z\}\}}

\pause \vspace{20pt} Simple expressions are also referred to as {\em terms}.
\end{frame}

\begin{frame}{patterns}

  A {\em pattern} is a syntactical construct that uses almost the same
  syntax as terms.

  \pause
  \vspace{20pt}

\begin{grammar}
<pattern> ::= <literal> 
      \alt <variable> 
      \alt '\_' 
      \alt '\{' <pattern> ',' <pattern> '\}'
\end{grammar}

 \pause \vspace{20pt}
  The \_ symbol can be read as ``don't care''.

\end{frame}





\begin{frame}{sequence}

\begin{grammar}
  <match> ::=  <pattern> '=' <expression>
\end{grammar}

\pause
\begin{grammar}
  <sequence> ::=  <expression> \alt <match> ',' <sequence>
\end{grammar}

\pause\vspace{20pt}

examples:
 \begin{itemize}
   \pause \item {\tt X = a, \{b, X\}}
   \pause \item {\tt X = a, Y = \{b, X\}, \{a, Y\}}
 \end{itemize}

\end{frame}

\begin{frame}{evaluation}

A sequence is evaluated given an {\em environment}, written $\sigma$ (sigma).

\pause\vspace{20pt}
The environment holds a set of variable substitutions (bindings):
$v/s \in \sigma$, $v$ is a variable and $s$ is a structure.

\pause\vspace{20pt} 
An evaluation of a sequence $e$ given an environment
$\sigma$ is written $E\sigma(e)$. 

\pause\vspace{20pt}
We write:
\vspace{20pt}
$$E\sigma(e) \rightarrow s$$
\vspace{20pt}
where $s$ is a {\em data structure}.

\end{frame}

\begin{frame}{data structures}
 
  We have a small domain, consisting only of atoms and a binary compound
  structures.
  \pause
  \vspace{20pt}

  \begin{tabular}{r l l}
   {\em Atoms} & = & \{a, b, c, \ldots\} \\
   {\em Structures} & = & {\em Atoms} $\cup$ \{ \{a, b\} \textbar a $\in$ {\em Structures}  $\wedge$  b $\in$ {\em Structures} \}
  \end{tabular}

  \pause \vspace{20pt}

  {\em These are the data structures that will be the result of our evaluation.}

  \pause \vspace{10pt}

  \begin{itemize}
     \pause \item {\em a}
     \pause \item {\em \{a, \{b, c\}\}}
  \end{itemize}

\end{frame}


\begin{frame}{equivalence relation}

We have a one-to-one mapping from pattern- and term-atoms, to atoms in the set of data structures.

\pause \vspace{20pt}

For every term atom {\tt a}, there is a corresponding atom structure $s$.

\pause \vspace{20pt}

{\em For every digit \texttt{1,2,3} (or \texttt{I, II, III}) there is a corresponding number $1,2,3$.}

\vspace{20pt}
We write ${\rm a} \equiv s$.

\end{frame}


\begin{frame}{evaluation of expressions}

We have the following rules for evaluation of expressions:

 \begin{itemize}
   \pause \item $E\sigma(a) \rightarrow s$ if $a \equiv s$ 

   \pause \item $E\sigma(v) \rightarrow s$ if $v/s \in \sigma$

   \pause \item $E\sigma(\lbrace e_1 , e_2\rbrace) \rightarrow \lbrace E\sigma(e_1), E\sigma(e_2) \rbrace$

 \end{itemize}

 \vspace{20pt}\pause What if we have $E\sigma(v)$ and $\ v/s \not\in \sigma$?

\pause
\vspace{20pt}\pause $E\sigma(v) \rightarrow  \perp $ if  $\ v/s \not\in \sigma$

\pause
 \vspace{20pt}\pause Given the expression {\tt \{2+2, 3+3\}}, do we evaluate {\tt 2+2} before {\tt 3+3}?

\end{frame}

\begin{frame}{evaluation of expressions}

 assume: $\sigma = \lbrace X/\lbrace a, b\rbrace\rbrace$ 

  \begin{eval}
    \pause $E\sigma(c)$ & $\rightarrow $ \pause $c$\\
    \pause $E\sigma(X)$ & $\rightarrow $ \pause $\lbrace a, b \rbrace$
  \end{eval}

  \vspace{20pt}\pause assume: $\sigma = \lbrace X/a, Y/b \rbrace$ 

  \pause \begin{eval}
    $E\sigma(\lbrace X, Y\rbrace)$ & \pause $\rightarrow \lbrace E\sigma(X), E\sigma(Y) \rbrace$\\
                  &\pause $\rightarrow \lbrace a , b \rbrace$
  \end{eval}
\end{frame}

\begin{frame}{pattern matching}

The result of evaluating a {\em pattern matching} is a an
environment.  We write: $$P\sigma(p,s) \rightarrow \theta$$ where
$\theta$ (theta) is the set of variable bindings we obtain.

\pause\vspace{10pt}

\begin{itemize}
  \pause \item $P\sigma(a,s) \rightarrow \sigma$  if  $a \equiv s$
  \pause \item $P\sigma(\_,s) \rightarrow \sigma$  
  \pause \item $P\sigma(v,s) \rightarrow \sigma$  if \pause $ v/s \in \sigma $
\end{itemize}

\pause\vspace{10pt}

{\em Note that the second argument of the pattern matching is a data structure.}


\end{frame}

\begin{frame}{pattern matching}

\pause What do we do with $P\sigma(v,s)$ when $v/t \not\in \sigma$?

\vspace{10pt}\pause 
\begin{itemize}
 \pause \item $P\sigma(v,s) \rightarrow \lbrace v/s \rbrace \cup \sigma$ if \pause $ v/t \not\in \sigma$
\end{itemize}

\pause\vspace{20pt} What do we do with $P\sigma(a,s)$ when $a \not\equiv s$?

\pause\vspace{20pt} What do we do with $P\sigma(v,s)$ when $v/t \in \sigma$,  but $t \not\equiv s$?

\vspace{10pt}\pause 
\begin{itemize}
  \pause \item $P\sigma(a, s) \rightarrow$ \pause $\  \mathrm{fail}\ $ if $a \not\equiv s$
  \pause \item $P\sigma(v, s) \rightarrow$ \pause $\  \mathrm{fail}\ $ if $(v/t \in \sigma) \wedge t \not\equiv s$
\end{itemize}


\end{frame}

\begin{frame}{pattern matching}

If the pattern is a compound pattern, \pause the components of the pattern are matched to their corresponding sub structures.

\pause

 \begin{itemize}
   \pause \item $P\sigma(\lbrace p_1, p_2 \rbrace , \lbrace s_1, s_2 \rbrace) \rightarrow \theta$ if
   \begin{itemize}
     \pause \item  $P\sigma(p_1 , s_1) \rightarrow \theta' \wedge$
     \pause \item  $P\theta'(p_2 , s_2) \rightarrow \theta$
   \end{itemize}
 \end{itemize}

\pause \vspace{10pt}
Note that the second part is evaluated in $\theta'$. 

\pause \vspace{10pt}Example: \{X,\{Y,X\}\} = $\{a, \{b,c\}\}$

\vspace{20pt}{\em Match a compund pattern with anyting but a compound structure will fail.}

\end{frame}

\begin{frame}{examples}

assume: $\sigma = \lbrace Y/b\rbrace$

\begin{itemize}
  \pause\item $P\sigma(X , a) \rightarrow $\pause $\ \lbrace X/a \rbrace  \cup \sigma$
  \pause\item $P\sigma(Y , b) \rightarrow $\pause $\ \sigma$
  \pause\item $P\sigma(Y , a) \rightarrow $\pause  $\ \mathrm{fail}$
  \pause\item $P\sigma(\lbrace Y, Y\rbrace , \lbrace a, b \rbrace) \rightarrow $\pause $\ \mathrm{fail}$
\end{itemize}

\end{frame}

\begin{frame}{pattern matching}

\pause Pattern matching can {\em fail}. 

\pause\vspace{20pt}{\em fail} is different from $\perp$

We will use failing to guide the program execution, more on this later.

\end{frame}

\begin{frame}{evaluation of sequences}

A sequence is evaluated one pattern matching expression after the other. 

\begin{itemize}

  \pause \item $E\sigma(p = e, {\rm sequence}) \rightarrow E\theta({\rm sequence})$ 
     \begin{itemize} 
       \pause \item if $E\sigma(e) \rightarrow s$
       \pause \item and $P\sigma(p, s) \rightarrow \theta$
     \end{itemize}
\end{itemize}

\pause
The evaluation of the final expression is the result of the evaluation of the sequence.

\end{frame}

\begin{frame}{example}

  evaluate {\tt X = a, Y = b, \{X,Y\}}

\pause 
\begin{eval}
$E\lbrace \rbrace(${\tt X = a, Y = b, \{X,Y\}}$)$ & $\rightarrow \pause E\lbrace X/a\rbrace(${\tt Y = b, \{X,Y\}}$)$\\
   &  $\rightarrow \pause E\lbrace Y/b, X/a\rbrace(${\tt \{X,Y\}}$) $\\
   &  $\rightarrow \pause \lbrace E\lbrace Y/b, X/a\rbrace(X), E\lbrace Y/b, X/a\rbrace(Y)\rbrace$\\
   &  $\rightarrow \pause \lbrace a, E\lbrace Y/b, X/a\rbrace(Y)\rbrace$\\
   &  $\rightarrow \pause \lbrace a, b \rbrace$
\end{eval}

\end{frame}



\begin{frame}{Where are we now}

We have defined the semantics of a programming language (not a
complete language) by defining how expressions are evaluated.

\vspace{20pt} 

\pause Important topics:

\vspace{10pt} 

\begin{itemize}
 \pause \item set of data structures: atoms and compound structures
 \pause \item environment: that binds variables to data structures
 \pause \item expressions: term expressions, pattern matching expressions and sequences
 \pause \item evaluation: from expressions to data structures $E\sigma(e) \rightarrow s$
\end{itemize}

\end{frame}

\begin{frame}{Why}

\vspace{40pt}\hspace{80pt}Why do we do this?

\end{frame}


\begin{frame}{more}
What is missing:
\pause
\begin{itemize}
  \item evaluation of {\em case} (and {\em if} expressions)
  \item evaluation of function applications
\end{itemize}
\end{frame}

\end{document}
