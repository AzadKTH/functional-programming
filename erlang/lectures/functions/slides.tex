\input{../include/preamble}

\title[ID1019 Evaluation of functions]{Evaluation - case and functions}


\author{Johan Montelius}
\institute{KTH}
\date{\semester}

\begin{document}

\begin{frame}
\titlepage
\end{frame}

\begin{frame}{recapitulation}

\begin{itemize}
 \pause \item defining the {\em operational semantics} of a language
 \pause \item a specification for an implementation of the language
 \pause \item a tool for us to talk about the language
\end{itemize}

\end{frame}

\begin{frame}{data structures}

  We have a small domain, consisting only of atoms and a binary compound
  structures.
  \pause
  \vspace{20pt}

  \begin{tabular}{r l l}
   {\em Atoms} & = & \{a, b, c, \ldots\} \\
   {\em Structures} & = & {\em Atoms} $\cup$ \{ \{a, b\} \textbar a $\in$ {\em Structures}  $\wedge$  a $\in$ {\em Structures} \}
  \end{tabular}

  \pause
  \vspace{20pt}
  {\em We talk about atomic and compound structures.}
\end{frame}

\begin{frame}{expressions}

\begin{grammar}
<atom> ::= a | b | c | \ldots

<variable> ::= X | Y | Z | \ldots

<literal> ::= <atom>

<expression> ::= <literal> | <variable> |  '\{' <expression> ',' <expression> '\}'

<pattern> ::= <literal> | <variable> | '\_' | '\{' <pattern> ',' <pattern> '\}'

<match> ::=  <pattern> '=' <expression>

<sequence> ::=  <expression> | <match> ',' <sequence>
\end{grammar}

\end{frame}

\begin{frame}{an environment}

\pause\vspace{20pt}
The environment holds a set of variable substitutions (bindings):
$v/s \in \sigma$, $v$ is a variable and $s$ is a structure.


\end{frame}


\begin{frame}{evaluation of expressions}

 \begin{itemize}
   \pause \item $E\sigma(a) \rightarrow s$ if $a \equiv s$

   \pause \item $E\sigma(v) \rightarrow s$ if $v/s \in \sigma$

   \pause \item $E\sigma(\lbrace e_1 , e_2\rbrace) \rightarrow \lbrace E\sigma(e_1), E\sigma(e_2) \rbrace$

 \end{itemize}

\pause\vspace{20pt}{\em Evaluation can result in $\perp$, if a variable is unbound.}

\end{frame}


\begin{frame}{pattern matching}

\begin{itemize}
  \pause \item $P\sigma(a, s) \rightarrow \sigma$  if  $a \equiv s$
  \pause \item $P\sigma(\_,s) \rightarrow \sigma$  
  \pause \item $P\sigma(v, s) \rightarrow \sigma$  if \pause $ v/s \in \sigma $
  \pause \item $P\sigma(v, s) \rightarrow \lbrace v/s \rbrace \cup \sigma$ if \pause $ v/t \not\in \sigma$
   \pause \item $P\sigma(\lbrace p_1, p_2 \rbrace  = \lbrace s_1, s_2 \rbrace) \rightarrow \theta$ if
   \begin{itemize}
     \pause \item  $P\sigma(p_1, s_1) \rightarrow \sigma' \wedge P\sigma'(p_2, s_2) \rightarrow \theta$
   \end{itemize}

\end{itemize}

\pause\vspace{20pt}
Pattern matching can {\em fail}. 

\end{frame}

\begin{frame}{evaluation of sequences}

\begin{itemize}
  \pause \item $E\sigma(p = e, {\rm sequence}) \rightarrow E\sigma'({\rm sequence})$ 
     \begin{itemize} 
       \pause \item if $E\sigma(e) \rightarrow s$
       \pause \item and $P\sigma(p, s) \rightarrow \sigma'$
     \end{itemize}
\end{itemize}

\pause\vspace{20pt}{\em First evaluate the pattern matching expression, if it succeeds then continue with the new environment.}
\end{frame}

%%--------------


\begin{frame}[fragile]{case expression}

\begin{verbatim}
   case  X  of 
       a ->  foo;
       b ->  bar
   end
\end{verbatim}

\end{frame}

\begin{frame}{case expression}

\begin{grammar}
     <expression> ::=  <case expression> | ...  

     <case expression> ::= 'case' <expression> 'of' <clauses>  'end' 

     <clauses> ::=   <clause> | <clause> ';' <clauses>

     <clause> ::=  <pattern> '->' <sequence>
\end{grammar}
\end{frame}

\begin{frame}{evaluation of case expression}
 
  \begin{itemize}
  \item $E\sigma({\tt case}\ e \ {\tt of}\ {\rm clauses} \ {\tt end}) \rightarrow C\sigma(s, {\rm clauses})$  \pause 
    \begin{itemize}
    \item if $E\sigma(e) \rightarrow s$
    \end{itemize}
       
  \end{itemize}


 \vspace{20pt}\pause $C\sigma(s, {\rm clauses})$ will select one of
 the clauses based on the patterns of the clauses and then continue the
 evaluation of the sequence of the selected clause.
\end{frame}

\begin{frame}{selection of a clause}

  \begin{itemize}
     \pause \item $C\sigma(s, p \rightarrow  {\rm sequence} ; {\rm clauses}) \rightarrow E\sigma'({\rm sequence})$ 
     \begin{itemize}      
        \pause\item if $P\sigma(p, s) \rightarrow \sigma'$
        \pause\item otherwise \ldots
     \end{itemize}
     \pause \item $C\sigma(s, {\rm clause} ; {\rm clauses}) \rightarrow C\sigma(s, {\rm clauses})$
  \end{itemize}

Similar for the last clause of the clauses. 

\pause \vspace{20pt}
If no successful clause is found, the evaluation is undefined and returns $\perp$.

\end{frame}
\begin{frame}{example}

\begin{eval}
  \pause$E\lbrace X/\lbrace a,b\rbrace\rbrace(${\tt case X of a -> a; \{\_,Y\} -> Y end}$) \rightarrow$ 
\end{eval}

\begin{eval}
   \hspace{40pt}\pause$E\lbrace X/\lbrace a,b\rbrace\rbrace(${\tt X}$) \rightarrow \lbrace a,b\rbrace$
\end{eval}

\begin{eval}
   \pause $C\lbrace X/\lbrace a,b\rbrace\rbrace(\lbrace a,b\rbrace, ${\tt \ a -> a; \{\_,Y\} -> Y}$) \rightarrow$ 
\end{eval}
\begin{eval}
   \hspace{40pt}\pause $P\lbrace X/\lbrace a,b\rbrace\rbrace( a, \lbrace a,b\rbrace) \rightarrow \mathrm{fail}$
\end{eval}

\begin{eval}
   \pause $C\lbrace X/\lbrace a,b\rbrace\rbrace(\lbrace a,b\rbrace, ${\tt \  \{\_,Y\} -> Y}$) \rightarrow$ 
\end{eval}
\begin{eval}
   \hspace{40pt}\pause $P\lbrace X/\lbrace a,b\rbrace\rbrace(\{\_, Y\}, \lbrace a,b\rbrace) \rightarrow $ \pause $\lbrace Y/b,\  X/\lbrace a,b\rbrace\rbrace$
\end{eval}

\begin{eval}
  \pause$E\lbrace Y/b, X/\lbrace a,b\rbrace\rbrace(${\tt Y}$) \rightarrow $
\end{eval}
\begin{eval}
  \pause$b$
\end{eval}
  
\end{frame}


\begin{frame}{free variables}

Are all syntactical correct sequences also valid sequences?

\pause\vspace{20pt}

A sequence must not contain any {\em free variables}.

\pause\vspace{10pt}

A free variable in a sequence, {\tt <sequence>}, is bound by the
pattern matching expressions in the sequence {\tt <patter> =
<expression>, <sequence>} if the variable occurs in the {\tt
<pattern>}.

\pause\vspace{10pt}

A free variable in a , {\tt <sequence>}, is bound by the
pattern matching expressions in the clause {\tt <pattern> -> <sequence>} if the variable occurs in the {\tt
<pattern>}.


\end{frame}

\begin{frame}{free variables}

{\tt X = a, \{Y,Z\} = \{X,b\}, \{X,Y,Z\}}

\pause\vspace{20pt}

{\tt \{Y,Z\} = \{X,b\}, \{X,Y,Z\}}


\pause\vspace{20pt}


{\tt X = \{a,b\}, case X of \{a,Z\} -> Z end}


\end{frame}



\begin{frame}[fragile]{free variables}

\vspace{20pt}

\hspace{20pt}\begin{verbatim}
   X = {a,b},
   Y = case X of
         {a, Z} -> {c, Z}
       end,
   {Y, Z}
\end{verbatim}

\vspace{20pt}{\em This is not allowed, \texttt{Z} in  \texttt{\{Y,Z\}} is a free variable. However .... \pause this is allowed in Erlang (but it should not be).}

\end{frame}


 

\begin{frame}{function definitions}

We could do without named  function definitions  but it's convenient to include them in our language. 

\vspace{20pt}

Assume we have a {\em program}, $P$, that is a set of named function definitions on the form:

  $${\rm name}(v_1, ...) \rightarrow  {\rm sequence}$$

where $v_1, ...$ is a sequence of unique variables called the {\em parameters} of the function. 

\pause \vspace{20pt}

Any free variable in sequence of the definition must occur in the parameters.

\end{frame}

\begin{frame}{function application}

We extend the definition of expressions, to include function application:

\pause
\begin{grammar}
<expression> ::= <fun> '(' <arguments> ') | \ldots

<arguments> ::= <expression> | <expression> ',' <arguments>

<fun> ::= <atom>
\end{grammar}

\pause {\em We will later extend the definition of functions but let us keep things simple for now.}

\end{frame}


\begin{frame}{evaluation of function application}

Assume we have a program $P$ that consist of a set of named functions. 

 \begin{itemize} 
  \pause\item $E\sigma({\rm name}(e_1, \ldots)) \rightarrow E\{v_1/s_1, \ldots\}({\rm sequence} )$ 
   \begin{itemize}        
           \pause\item if $E\sigma(e_i) \rightarrow s_i$ and 
           \pause\item ${\rm name}(v_1, \ldots) \rightarrow  {\rm sequence}  \in P$ 
    \end{itemize} 
 \end{itemize} 

\pause\vspace{20pt}

{\em Note that the sequence is evaluated in an environment independent of $\sigma$.}

\end{frame}

\begin{frame}[fragile]{example}

\begin{verbatim}
append(X,Y) ->
    case X of 
       nil -> Y; 
       {H, T} -> {H, append(T, Y)} 
    end
\end{verbatim}

\pause\vspace{20pt}
evaluate $E\lbrace\rbrace( ${\tt append(\{a,\{b,nil\}\}, \{c,nil\})}$ )$

\end{frame}

\begin{frame}{example}

\begin{eval} 
\pause $E\lbrace\rbrace( ${\tt append(\{a,\{b,nil\}\}, \{c,nil\})}$ ) \rightarrow$\\
\pause\hspace{10pt} $\sigma_1 = \lbrace X/\lbrace a, \lbrace b, {\rm nil}\rbrace\rbrace, Y/\lbrace c, {\rm nil}\rbrace \rbrace$\\
\end{eval}

\vspace{10pt}

\begin{eval} 
\pause $E\sigma_1($ {\tt case X of nil -> Y; \{H,T\} -> \{H,append(T,Y)\} end} $) \rightarrow$\\
\pause\hspace{10pt}  $\sigma_2 = \lbrace H/a, T/\lbrace b, {\rm nil}\rbrace, X/\lbrace, a, \lbrace b, {\rm nil}\rbrace\rbrace, Y/\lbrace c, {\rm nil}\rbrace \rbrace$\\
\end{eval}

\vspace{10pt}

\begin{eval} 
\pause $E\sigma_2($ {\tt \{H, append(T, Y)\}} $) \rightarrow$\\
\pause $\lbrace E\sigma_2($ {\tt  H} $), E\sigma_2($ {\tt append(T, Y)} $) \rightarrow$\\
\pause $\lbrace a, \ E\sigma_2($ {\tt  append(T, Y)} $) \rbrace \rightarrow$\\
\pause\hspace{10pt} $\sigma_3 = \lbrace X/ \lbrace b, {\rm nil}\rbrace, Y/\lbrace c, {\rm nil}\rbrace \rbrace$\\
\end{eval}

\vspace{10pt}

\begin{eval} 
\pause $\lbrace a ,\  E\sigma_3(${\tt case X of nil -> Y; \{H,T\} -> \{H,append(T,Y)\} end} $)\rbrace \rightarrow$\\
\end{eval}

\end{frame}


\begin{frame}{example}

\begin{eval} 
\pause $\lbrace a,\ E\sigma_3(${\tt case X of nil -> Y; \{H,T\} -> \{H,append(T,Y)\} end} $)\rbrace \rightarrow$\\
\pause\hspace{10pt} $\sigma_4 = \lbrace H/b, T/{\rm nil}, X/\lbrace b, {\rm nil}\rbrace, Y/\lbrace c, {\rm nil}\rbrace \rbrace$\\ 
\end{eval}

\vspace{10pt}

\begin{eval} 
\pause $\lbrace a,\ E\sigma_4($ {\tt \{H, append(T,Y)\}} $) \rbrace \rightarrow$\\
\pause $\lbrace a,\ \lbrace b, \ E\sigma_4($ {\tt append(T,Y)} $)\rbrace \rbrace \rightarrow$\\
\pause\hspace{10pt} $\sigma_5 = \lbrace X/{\rm nil}, Y/\lbrace c, {\rm nil}\rbrace \rbrace$\\ 
\end{eval}

\vspace{10pt}

\begin{eval} 
\pause $\lbrace a,\ \lbrace b, \ E\sigma_5($ {\tt Y} $)\rbrace \rbrace \rightarrow$\\
\pause $\lbrace a,\ \lbrace b, \ \lbrace c, \ {\rm nil}\rbrace\rbrace$\\
\end{eval}

\end{frame}


\begin{frame}{much ado about nothing}

A lot of work for something that simple - why bother, it could not have
been done differently.
\end{frame}


\end{document}
