\input{../include/preamble}


\title[ID1019 Introduction]{Introduction}

\author{Johan Montelius}
\institute{KTH}
\date{\semester}

\begin{document}

\begin{frame}
\titlepage
\end{frame}

\begin{frame}{introduction}
  Functional programming - what?
\end{frame}


\begin{frame}{functional expressions}

$$(3 + 8) * (6 - 3)$$

\pause\vspace{20pt}

$$(\sqrt[3]{3 + 5^4})$$

\end{frame}

\begin{frame}{a small language}

\pause

\begin{itemize}
 \item a domain: $\mathbb{Z}$ i.e. ${... -2,-1,0,1,2... }$
\pause
 \item a set of primitive functions: $+$, $-$, $*$, $\mathrm{mod}$,  $\mathrm{div}$ 
\pause
 \item syntax: symbols, precedence, parentheses i.e. a way to write expressions
\end{itemize}

\end{frame}


\begin{frame}{evaluation of expressions}
\begin{columns}
 \begin{column}{0.3\linewidth}
  \begin{itemize}
   \pause \item $(3 + 5) * (6 - 3)$
   \pause \item $8 * (6 - 3)$
   \pause \item $8 * 3$
   \pause \item $24$
  \end{itemize}   
 \end{column}
 \begin{column}{0.3\linewidth}
  \begin{itemize}
   \pause \item $(3 + 5) * (6 - 3)$
   \pause \item $(3 + 5) * 3$
   \pause \item $8 * 3$
   \pause \item $24$
  \end{itemize}   
 \end{column}
 \begin{column}{0.3\linewidth}
  \begin{itemize}
   \pause \item $(3 + 5) * (6 - 3)$
   \pause \item $(3 + 5) * 3$
   \pause \item $(9 * 15)$
   \pause \item $24$
  \end{itemize}   
 \end{column}
 \end{columns}
\end{frame}

\begin{frame}{how about this}

\pause
$$5 * (4+2)$$

\pause
$$17 \ \mathrm{mod}\  5$$

\pause

$$7 \ \mathrm{mod}\  0$$

\end{frame}

\begin{frame}{bottoms}

\pause

\vspace{20pt}\hspace{60pt}$ 5 \ \mathrm{mod}\ 0 \equiv \perp$

\pause

\vspace{20pt}\hspace{60pt}$\perp$ is called {\em bottoms}, {\em undefined} or ... {\em exception}
\pause

\vspace{20pt}\hspace{60pt}We extend the domain: $\mathbb{Z} \cup \{\perp\}$
\pause 

\vspace{20pt}\hspace{60pt}How should we interpret: $5\ * \perp$

\end{frame}

\begin{frame}{strict functions}

\vspace{20pt}\hspace{80pt}\parbox[l][60pt][l]{240pt}{

A function that is defined to be $\perp$ if any of its arguments is $\perp$, is called a {\em strict function},

\vspace{20pt}All of our regular arithmetic functions are strict.
}

\end{frame}

\begin{frame}{ok, I get it}


\vspace{20pt}\hspace{40pt}What is the value of: $(x - x) * 5$

\end{frame}

\begin{frame}{evaluation of expressions}
\begin{columns}
 \begin{column}{0.5\linewidth}
  \begin{itemize}
   \pause \item $(\sqrt[3]{3 + 5^4}) * (6 - 6)$
   \pause \item $(\sqrt[3]{3 + 5^4}) * 0$
   \pause \item $0$
  \end{itemize}   
 \end{column}
 \begin{column}{0.5\linewidth}
  \begin{itemize}
   \pause \item $(512\ \mathrm{div}\ 0) * (6 - 6)$
   \pause \item $(512\ \mathrm{div}\ 0) * 0$
   \pause \item $0$
   \pause \item hmmm, not so good
  \end{itemize}   
 \end{column}
\end{columns}
\end{frame}

\begin{frame}{order of evaluation}

\vspace{20pt}\hspace{60pt}\parbox[l][60pt][l]{300pt}{If all functions are strict:
\begin{itemize}
\pause \item then all arguments of the function must be evaluated
\pause \item the order does not matter\pause ,... or does it?
\end{itemize}}

\end{frame}

\begin{frame}{if-then-else}

  Assume we have a function $if(test, then, else)$ with the obvious definition.

  \pause \vspace{20pt}

  Do we want this function to be a {\em strict function}?
  
\end{frame}


\begin{frame}{variables and functions}


Too make life more interesting, we introduce 

\vspace{10pt}\hspace{40pt}variables: $x$, $y$, 

\vspace{10pt}\hspace{40pt}and functions: $\lambda x \rightarrow x+5$

\pause

\vspace{20pt}{\em Most often written $\lambda x.x+5$ but we will use $\rightarrow$.}

\pause

\vspace{20pt}{\em So far, functions do not have names.}

\end{frame}


\begin{frame}{functions}
  \begin{itemize}
   \pause \item $\lambda x \rightarrow x + 5$
   \pause \item $(\lambda x \rightarrow x + 5) \ 7$
   \pause \item $(7 + 5)$
   \pause \item $12$
  \end{itemize}
\end{frame}

\begin{frame}{application}

We {\em apply a function} to an argument (or {\em actual arguments}),
\vspace{10pt}

  \begin{itemize}
   \pause \item $(\lambda x \rightarrow \langle E \rangle) 7$
  \end{itemize}

\pause\vspace{10pt}
by {\em substituting} the parameter (or {\em formal argument}) of the function with the argument.

\pause\vspace{10pt}
\begin{itemize}
  \item $[x/7]\langle E \rangle$
\end{itemize}
\end{frame}

\begin{frame}{examples}

\begin{itemize}
   \pause \item $[x/7]\langle x + 5 \rangle$  \hspace{20pt} \pause $7 + 5$
   \pause \item $[x/7]\langle \lambda y \rightarrow y + x \rangle$   \hspace{20pt} \pause $\lambda y \rightarrow y + 7$
   \pause \item $[x/(\lambda z \rightarrow z + 2)]\langle \lambda y \rightarrow (x y) * 2 \rangle$ 
\hspace{20pt} \pause  $\lambda y \rightarrow ((\lambda z \rightarrow z + 2) y) * 2$ 
 \end{itemize}

\pause\vspace{20pt}

But, things could go wrong.

\end{frame}

\begin{frame}{scope of declaration}

\pause

  In an expression $\lambda x \rightarrow  \langle E \rangle$, the {\em scope} of $x$
  is $\langle E \rangle$. 

\pause\vspace{10pt}

  We say that $x$ is {\em free} in $\langle E \rangle$ but {\em bound} in
  $\lambda x \rightarrow \langle E \rangle$.

\pause\vspace{60pt}

{\em We can write $\lambda x \rightarrow ( \lambda x -> (x*x))$, which does complicate things.}

\end{frame}


\begin{frame}{substitution}
  

A substitution $[x/\langle F \rangle]\langle E \rangle$ is possible if
$\langle F \rangle$ does not have any free variables ...
\pause

\vspace{5pt}\hspace{40pt} that become bound in $[x/\langle F \rangle]\langle E \rangle$.

\pause\vspace{20pt}

\begin{columns}
 \begin{column}{0.5\linewidth}
$$(\lambda x \rightarrow (\lambda y \rightarrow  (y + x))) (y + 5)$$
\pause
$$[x/(y + 5)] (\lambda y \rightarrow  (y + x))$$
\pause
$$\lambda y \rightarrow  (y + (y + 5))$$
 \end{column}

\pause
 \begin{column}{0.5\linewidth}
$$(\lambda x \rightarrow (\lambda z \rightarrow  (z + x))) (y + 5)$$
\pause
$$[x/(y + 5)] (\lambda z \rightarrow  (z + x))$$
\pause
$$\lambda z \rightarrow  (z + (y + 5))$$
 \end{column}

\end{columns}

\pause

\vspace{20pt}{\em We have to be careful but renaming variables solves the problem.}


\end{frame}



\begin{frame}{functions}
  A function is:
 \vspace{40pt}
 \begin{columns}[t]
      \begin{column}{0.5\textwidth}
        {\em \ldots a many to one mapping from one domain to another:  $A \mapsto B$ }
      \end{column}
      \begin{column}{0.5\textwidth}
        {\em \ldots a description of the expression that should be evaluated: $\lambda x \rightarrow x + 2$}
      \end{column}
   \end{columns}

\vspace{40pt}{\em In mathematics we can work with functions even if we do not know how to compute them.}


\end{frame}


\begin{frame}{$\lambda$ calculus}
  \begin{itemize}

    \pause \item The $\lambda$ calculus was introduced in the 1930s by Alonzo Church. 

    \pause \item Easy to define:
      \begin{itemize} 
         \pause \item only three types of expressions: variable, lambda abstraction, application 
         \pause \item only one rule: evaluation of application
         \pause \item you don't even need data structures nor named functions
      \end{itemize}

    \pause \item Anything that is {\em computable} can be expressed in
    $\lambda$ calculus, it is as powerful as a {\em Turing machine}.

    \pause \item We will use some extensions to the language when we describe
    functional programming.
  \end{itemize}
\end{frame}


\begin{frame}{currying}

  A function of two arguments, can be described as function of one
  argument that evaluates to another function of a second argument.

  \begin{itemize}
   \pause \item $(\lambda x \rightarrow (\lambda y \rightarrow x + y)) \ 7 \ 8$
   \pause \item $(\lambda y \rightarrow 7 + y) \ 8$
   \pause \item $7 + 8$
  \end{itemize}

 \vspace{10pt}
 We can write:
  \begin{itemize}
   \pause \item $\lambda x y \rightarrow  x + y$
  \end{itemize}

\end{frame}

\begin{frame}{let expressions}
  \begin{itemize}  
   \pause \item $\lambda x \rightarrow  (x + 2) +  (x + 2)$ \pause \hspace{20pt} do we have to evaluate $(x+2)$ twice?
   \pause \item $\lambda x \rightarrow  ((\lambda y \rightarrow y + y) (x + 2))$  \pause  \hspace{20pt} $(x+2)$ only evaluated once
   \pause \item $\lambda x \rightarrow  \LET y = x + 2 \IN y + y $  \pause  \hspace{20pt} more readable 
  \end{itemize}
\end{frame}

\begin{frame}{no recursive definitions}
  \begin{itemize}  
   \pause \item $\lambda x \rightarrow  \LET y = x + y \IN y + y $ \pause \hspace{20pt} What does this mean?
   \pause \item $\lambda x \rightarrow  ((\lambda y \rightarrow y + y) (x + y))$  
  \end{itemize}
\end{frame}

\begin{frame}{this is ok}
  \begin{itemize}  
   \pause \item $\lambda x \rightarrow  \LET y = x + 2 , z = y + 5 \IN z + z $
   \pause \item $\lambda x \rightarrow  ((\lambda y \rightarrow (\lambda z \rightarrow z + z)(y + 5)) (x + 2))$  
  \end{itemize}
\end{frame}


\begin{frame}{functional programming languages}

 \begin{itemize}
   \pause\item $\lambda$-calculus
       \begin{itemize}        
          \pause \item not the best syntax - not important 
          \pause \item no ``data structures'' - functions are all you need 
          \pause \item no need for named named functions
          \pause \item no defined evaluation order  
       \end{itemize}   
    \pause\item functional programming languages: 
       \begin{itemize}        
            \pause \item different syntax, some good some strange 
            \pause \item almost always provide built-in or user defined data structures 
            \pause \item named function i.e. the program
            \pause \item defines the evaluation order
       \end{itemize}               
 \end{itemize}

\pause\vspace{10pt}{\em All functional programming languages have a core that can be expressed in $\lambda$-calculus.}

\end{frame}

\begin{frame}{Erlang}

 \begin{itemize}
    \pause \item a syntax that is similar to the one used in for example Prolog 

    \pause \item a small set of built-in data structures, no user defined 

    \pause \item an ``eager evaluation'' order i.e. arguments are evaluated before the function is applied 
 \end{itemize}

\pause\vspace{40pt}
{\em Erlang is extended to be able to model concurrency. In the first part of this course we will only use the functional subset.}

\end{frame}


\begin{frame}{lambda expression}
  $\lambda x \rightarrow 2 + x$
  \pause

  \begin{code}
   fun(X) -> 2 + X end
  \end{code}

  \begin{code}
    (fun(Y) -> 2 + Y end)(4)
  \end{code}
\end{frame}


\begin{frame}{let expression}
  $\LET x = 2, y = x + 3 \IN y + y $
  \pause

  \begin{code}
    X = 2, Y = X + 3, Y + Y
  \end{code}
\end{frame}


\begin{frame}{function definition}

  $inc \equiv \lambda x \rightarrow x + 1$
  \pause

  \begin{code}
    inc(X) -> X + 1.
  \end{code}

\end{frame}

\begin{frame}{multiple arguments}

  $add \equiv \lambda x y \rightarrow x + y$
  \pause

  \begin{code}
    add(X, Y) -> X + Y.
  \end{code}

\end{frame}

\begin{frame}{literals}
 \begin{itemize}
  \pause \item {atoms:} {\tt a, b, foo, bar}
  \pause \item {integer:} {\tt 1,23, 3456789012345678901233456}
  \pause \item {bool:} {\tt true, false}
  \pause \item {more:} there are more but this is enough for now
 \end{itemize}
\end{frame}


\begin{frame}{compound structures}
 \begin{itemize}
  \pause \item {tuples:} {\tt \{\}, \{foo, bar\}, \{a, 42, true\}}
  \pause \item {lists:} [], [1|[]], [1,2,3,4]  (more on this later)
 \end{itemize}
\end{frame}

\begin{frame}{evaluation}

\pause
\vspace{100pt}\hspace{100pt}Next time.

\end{frame}


\end{document}
