\documentclass[a4paper,dvips,11pt]{article}

\usepackage{pstricks}
\usepackage{pst-node}
\usepackage{pst-tree}
\usepackage{wrapfig}
\usepackage{ifthen}
\usepackage[latin1]{inputenc}

\usepackage[dvips,%
            breaklinks=true,
            bookmarksnumbered=true,
            pdftitle={Distributed Computing},
            pdfauthor={Johan Montelius},
            pdfsubject={playing around with ICY},
            pdfkeywords={distributed computing media distribution erlang}
            ]{hyperref}

\newcommand{\nnsection}[1]{
\section*{#1}
\addcontentsline{toc}{section}{#1}
}

\begin{document}

\begin{center}
\vspace{20pt}
\textbf{\large A Shoutcast Jukebox }\\
\vspace{10pt}
\textbf{Johan Montelius}\\
\vspace{10pt}
\today{}
\end{center}

\nnsection{Introduction}

In this assignment you will build a small jukebox. We will play around
with shoutcast streams and build a shoutcast server. You will use the
Erlang bit-syntax to implement a communication protocol over HTTP. The
parser will be implemented using higher order functions to hide the
socket interface. You will learn how to decode a mp3 audio stream and
make it available for connecting media players. Sounds fun? - Let's
go!

\section{Shoutcast - ICY}

You first need to understand how shoutcast works. It's very simple
protocol for streaming audio content built using HTTP. It was at first
called ``I Can Yell'' and you therefore see a lot of {\tt icy} tags in
the HTTP header; we will also refer to it as the ICY protocol. 

One module, {\tt icy}, will implement all details of how the ICY
protocol is encoded and decoded. To make it more interesting we will
use higher order functions and make the module both able to handle
incomplete sequences and being unaware of how byte sequences are
actually read or written.

\subsection{request - reply}

A media client, such as Amarok or VLC, connects to a server by sending a HTTP
GET request. In this request the client asks for a specific feed in the
same way as it would ask for a web page. In the request header it also
announce if it can handle meta-data, the name of the player etc. A
request could look like this:


\begin{verbatim}
GET / HTTP/1.0<cr><lf>
Host: mp3-vr-128.smgradio.com<cr><lf>
User-Agent: Casty<cr><lf>
Icy-MetaData: 1<cr><lf>
<cr><lf>
\end{verbatim}


The {\tt Icy-Metadata} header is important since it signals that our
client is capable of receiving meta data in an audio stream. To see that
this actually works you could use {\tt wget} and look at what for
example Virgin Radio returns when you try to connect. Try the
following command in a Unix shell, but terminate it with ctrl-c or it will
keep reading the audio stream.

\begin{verbatim}
wget --header="Icy-MetaData:1" -S -O body.txt \
  http://mp3-vr-128.smgradio.com
\end{verbatim}

Now look at the reply (you did terminate the loading right). Decode
the reply and try to figure out what the different header tag
means. One reply header that is very important for us is the {\tt
  icy-metaint} header. It typically has the value 8192 (or 4096) which
is the number of bytes in each mp3-block that is sent. Since the bit
rate for this audio stream is 128Kb/s a 8192 byte block will contain
half a second of music. Look at position 8192 in the body (body.txt),
do you find the meta data?


\subsection{meta-data}

The meta data comes as a sequence of characters after each audio
block. The length of the sequence is coded in one integer $k$ in the
first byte of the meta data block. The length of the sequence is
$16*k$ bytes (not including the k-byte); the smallest meta data block
is thus simply one k-byte of zero. If the text message is not an even
multiple of $16$ it is padded with trailing zeros. In Erlang
bit-syntax (do some reading on bit-syntax, you will need it) a meta
data segment could be written as follows:

\begin{verbatim}
<<1,104,
  101,108,
  108,111,
  0,0,0,0,
  0,0,0,0,
  0,0,0>>
\end{verbatim}

The first byte is a the length of the following padded string. The
next five bytes spell out ``hello'' and the padding consist of 11
zeros. This could also be written:

\begin{verbatim}
   <<1,"hello", 0:(11*8)>>
\end{verbatim}

When you attach to a shoutcast radio station you will see that most
meta data blocks are empty. When a new song i played they use the meta
data to send the name of the artist, title etc. We could for example
use this string to encode the title:

\begin{verbatim}
   "StreamTitle:'This is the title';"
\end{verbatim}

If this string is padded correctly it will be used by the client to
display the title of the current song.


\subsection{decoding a request}

We will of course communicate using sockets but why build this into
the icy module. The icy module will only deal with the format of
request and replies and not how these are actually read or written.

\subsection{continuations}

To understand the structure of our parser one must understand the
problem with reading from a stream socket. Reading from a socket is
very different from reading from a file. When reading from a socket
there are two things we need to handle, what happens if there is no
data to read and, what if only half of the data is available. 

If there is no data to read we will be suspended until the client
closes the connection and in a concurrent system this might be a
problem. If a process is suspended in a read operation we will not be
able to send it any messages (for example to quit what it is doing).

In Erlang there is a solution to this problem. Instead of ``reading''
from the socket, we have the socket send us segments as they
arrive. We can thus go into a receive statement and wait for either a
segment or any other message. Read about how sockets can be opened in
{\tt active} mode.

A process that is receiving segments from socket
could then be programmed as follows:

\begin{verbatim}
reader(Socket) ->
   receive
       {tcp, Socket, Segment} ->
           icy:parse_request(Segment);
       {tcp_closed, Socket} ->
           {error, "socket closed"};
       stop ->
           ok
   end.
\end{verbatim}

The process that calls {\tt reader/1} can now be aborted by sending a
stop message. This solves the first problem but the second remains,
the received segment could contain only half a message. The parser
must be aware of this and inform us that more data is needed. The
structure could then be as follows:

\begin{verbatim}
reader(Socket, Sofar) ->
   receive
       {tcp, Socket, Rest} ->
           Segment = <<Sofar/binary,Rest/binary>>,
           case icy:parse_request(Segment) of
              {ok, Parsed} -> 
                   Parsed;
              more ->
                   reader(Socket, Segment)
           end;
       {tcp_closed, Socket} ->
           {error, "socket closed"};
       stop ->
           ok
   end.
\end{verbatim}

This has the drawback that we might have to parse the same message
several times if it broken up into several segments. We need to try to
parse it because there is no other way of telling if we have a complete
message, but the problem is that we have to start from the beginning
every time.


To make it more efficient we can use something called
``continuations''. The parser will then return either the parse tree
or a function that we can use to continue the parsing. 

\begin{itemize}

 \item {\tt \{ok, Parsed, Rest\}} : if the parsing was successful,
   Parsed is the result of the parser and Rest is what is left of the
   segment.

 \item {\tt \{more, Continuation\}} : if more segments are needed,
   Continuation is a function that should be applied to the next segment.

 \item {\tt \{error, Error\}} : if the parsing failed.

\end{itemize}

We then define a general purpose user of the parser that uses a zero
argument function. The function is applied without arguments and
could then result in a parsed result, a request for more or an error. 

\begin{verbatim}
reader(Parser, Socket) ->
    receive
       {tcp, Socket, More} ->
           case Parser(More) of
               {ok, Parsed, Rest} ->
                   {ok, Parsed, Rest};
               {more, Cont} ->
                    reader(Cont,  Socket);
               {error, Error} ->
                    {error, Error}
            end;
       {tcp_closed, Socket} ->
            {error, "server closed connection"};
       stop ->
             ok
    end.
\end{verbatim}

If we receive a new TCP block we construct a new function and apply
{\tt reader/2} recursively. To read and parse a message from a socket,
we could call the reader as follows:

\begin{verbatim}
reader(Socket) ->
    reader(fun(More) -> icy:parse_request(More) end, Socket).
\end{verbatim}

The user of the parser will be a processes that we will define
later. The server will accept request from a client, Amarok or VLC, and
start streaming music.

\subsection{the parser}

So now let's look at the parser. We will only need to parse one type
of messages: the request. The request will be sent by a media client
to our server and the server will reply with a shoutcast stream. The
socket will deliver information in binary format since delivering the
audio stream in binary format is more efficient. The parser of course
then has to parse binary data but with the bit-syntax this is very
similar to parsing from a list of characters.


\subsubsection{a request}

Parsing a request is quite simple, we read the first line (all lines
are terminated by \textless cr\textgreater \textless lf\textgreater)
and check that it is equal to ``GET / HTTP/1.1''. Now this is of
course a simplification; a request could of course include something
more interesting than ``/'' (the resource we are asking for) but it
will do for now.

The {\tt request\_line/2} function will try to parse a complete request-line, if this
is not possible it will return {\tt more}. The request parser
will then return a continuation in the form of a function that should
be applied to the next segment. We could make this more complicated
but why not simply return a function that appends the segment that we
have to the next segment and tries to redo the parsing of the request.

\begin{verbatim}
parse_request(Segment) ->
  case request_line(Segment) of
    {ok, Method, Resource, Version, R1} ->
      case parse_headers(R1, []) of
        {ok, Headers, Body} ->
           {ok, {Method, Resource, Version, Headers}, Body};
         more ->
           {more, fun(More) -> parse_request(<<Segment/binary, More/binary>>) end}
      end;
    {ok, Req, _} ->
      {error, "invalid request: " ++ Req};
    more ->
      {more, fun(More) -> parse_request(<<Segment/binary, More/binary>>) end}
  end.
\end{verbatim}

Note how the scoping rules work; Erlang uses lexical scoping and the
variable {\tt Segment} will have it's binding when the function {\tt
  request/1} is called. The returned function could now be called in
any environment and the {\tt Segment} variable will maintain it's
binding. This is called static or lexical scoping. 

Once we have seen the request line we continue to parse the
headers. The parsing of the headers will either succeed or result in a
request of more information. If we succeed we return the Headers and
also what is left of the segment that we parsed. In practice the rest
will be an empty segment and probably ignored by the caller but why
not be polite and return what is left.

Note that we have made a simplification in the implementation of the
parse. If the function {\tt request\_line/1} or {\tt parse\_headers/2} has stepped
through a hundred characters only to find out that more characters are
needed, they simply return the atom {\tt more}. The next time they are
called they will have to run through the same hundred characters
again. It would of course be nice if we could save exactly the
position that we're in and continue only with the new segment. This
would however make the function {\tt parse\_request/1} more complicated. We
would have to keep track of if we should continue reading a line or a
header. 

Reading a line is a simple pattern matching exercise that is left as
an exercise. This is skeleton to work on: (\textless cr\textgreater
could be written as {\tt 13} or {\tt \$\textbackslash r} and \textless
lf\textgreater as {\tt 10} or {\tt \$\textbackslash n}). 

\begin{verbatim}
parse_request_line(Segment) when size(Segment) < 4 ->
    more;
prse_request_line(<<$G, $E, $T, 32, R1/binary>>)     ->
    case parse_resource(R1) of
        {ok, Resource, R2} ->
           case parse_version(R2) of 
               {ok, Version, R3} ->
                   {ok, get, Resource, Version, R3};
               more ->
                   more;
               Other ->
                   {error, "strange version: " ++ Other}
           end;
        more ->
            more;
        error ->
            {error, "failed to parse resource"}
    end;
request_line(Line) ->
    {error, "not a GET request: " ++ Line}.
\end{verbatim}

To parse the resource you simply scan the sequence until you find a
space character. Everything up to the space character is the resource
we're looking for.


When you implement {\tt parse\_version/1} you need to identify
when/if you need more data. You can check if the segment contains at
least ten characters using a guard statement. 

\begin{verbatim}
parse_version(Segment) when size(Segment) <  10 ->
    more;
parse_version(<<..., .., ..., Rest/binary>>) ->
                   {ok, v10, Rest};
parse_version(<<"..., .., ...,  Rest/binary>>) ->
                   {ok, v11, Rest}.
\end{verbatim}

Parsing the headers should be equally simple. We read headers until we
hit a empty line (\textless cr \textgreater, (\textless lf
\textgreater). The individual headers can be represented by tuples
            {\tt \{Key, Value\}}, where key is an atom and the value
            is a string.  Since we will not look into these headers
            you could also simply represent them as a single string
            i.e. {\tt ``foo: 12''} instead of {\tt\{foo, ``12''\}}.

\subsection{does it work}

Complete a module {\tt icy} that exports the above described
functions. You can then test your implementation with the following test examples:

\begin{verbatim}
icy:parse_request(<<"GET / HTTP/1.1\r\nkey:value\r\n\r\n">>).
\end{verbatim}

Let's try parsing something incomplete.

\begin{verbatim}
{more, F} = icy:parse_request(<<"GET / HTTP/1.0\r\nkey:value\r\n">>).
\end{verbatim}
The request is incomplete since we need an extra \verb+\r\n+. We do now
however, have a closure that we can apply to some additional information.

\begin{verbatim}
F(<<"\r\n">>).
\end{verbatim}

This should hopefully give us the parsed result. Identify the
different pieces of information, the request type, the resource
requested, the headers, the version and, the body.

\begin{verbatim}
{ok,{get,"/",[{key,"value"}],v10},<<>>}
\end{verbatim}



\subsection{encoding the stream}

The last thing we need to implement in the icy module is how to encode
the reply to the client. We need to encode the HTTP status line and
headers that should be the first thing we reply. We also need to
encode mp3 content in a sequence of segments of equal size separated
by meta data.

The encoding of the HTTP reply could look as follows. The most
important thing is that we supply the {\tt icy-metaint} header that
describes the size of our segments. Other headers can be added that
tells the media client a bit about our radio station etc.

\begin{verbatim}
encode_response(Header) ->
    Status = "ICY 200 OK\r\n",
    MetaInt = "icy-metaint: " ++ integer_to_list(?Chunk) ++ "\r\n",
    Reply = Status ++ MetaInt ++ encode_header(Header),
    list_to_binary(Reply).
\end{verbatim}

A server that replies to a client should first send a HTTP reply as
shown above and then send a stream och segments and meta data. The
segments is just a chunk of mp3 data. The only problem we have when
dividing some mp3 data into chunks is what to do with the last chunk
if it does not fill a full segment. If we are a radio station we would
of course just append the next song but we will simply throw the last
chunk away. This might sound stupid but it will make things a lot
easier when we extend this jukebox.

\begin{verbatim}
segments(<<Chunk:?Chunk/binary, Rest/binary>>) ->
    [{seg, Chunk}| segments(Rest)];
segments(_) ->
    [].
\end{verbatim}

The meta data is a bit tricky: it needs to be padded with zeros, it
should be of a length multiple of 16 plus one and the first byte
should state how many multiples we have. The content itself should be
a string of tags and arguments as you have seen before. The tricky
thing is how to pad this string, you will have to do some modular
arithmetic to get it right.

\begin{verbatim}
encode_meta(Meta) ->
    Encoded = encode_meta_info(Meta),
    {K, Padded} = padding(Encoded),
    <<K/integer, Padded/binary>>.    
\end{verbatim}

Assume that the meta data is given as a list of tag value tuples, for
example {\tt \{title, ``this is the title''\}}. You need to encode
each one of them as a string, for example {\tt ``StreamTitle='this is
  the title';''}, append them all together and then do the
padding. Note that the information elements are all encoded in one
sequence, separated by semicolon.


\section{Something about mp3}

It turns out that we do not need to know very much about mp3 encoding;
we will simply read the encoded data from a file and send it as it is
to the media client. There is however one thing that we need to know,
ID3 tags. The ID3 tags are not part of the mp3 standards (or MPEG),
it's a hack added later to encode the title, artist, track number etc
in the mp3 file. We need to find this tag, extract it and interpret
it.

\subsection{ID3 tags}

There is a whole family of ID3 tags but we will only look at the
original. It consists of the last 128 bytes in a file and has the
following format.

\begin{table}[center]
\begin{tabular}{|l|l|}
``TAG'' & 3 bytes, the ASCII value of TAG, works as an indicator\\
title& 30 bytes\\
artist& 30 bytes\\
album& 30 bytes\\
year& 4 bytes\\
comment& 28 bytes\\
0& 1 byte, a delimiter in version 1.1\\
track & 1 byte \\
genre & 1 byte \\
\end{tabular}
\caption{The format of a ID3 tag}
\end{table}

Create a module, mp3, and define the following procedure. You need to
write the {\tt id3\_tag/2} function that looks at the last 128 bytes
of a file (you can read at a specified position using {\tt
  file:pread(S, Size-128, 128)}) and determines if we have a ID3
tag. 

\begin{verbatim}
read_file(File) ->
    Size = filelib:file_size(File),
    {ok, S} = file:open(File, [read, binary, raw]),
    {ok, Id3, End} = id3_tag(S, Size),
    {ok, Data} = file:pread(S, 0, End),
    Title = id3_title(Id3),
    {mp3, Title, Data}.
\end{verbatim}

If there is a tag you should parse it and return a tuple on the
format below. Note that the strings are all coded as 30 byte
sequences where the last bytes have been set to zero. You need to
remove those trailing zeros in order to get a printable string.

\begin{verbatim}
{ok, {id3,[{title, "This is the title"}
           {artist, "Bobby Womak"}
                 :
                 :
          ]}
      645677}
\end{verbatim}

If no ID3 tag is found you should return {\tt \{ok, na, Size\}}. The
function call thus always succeeds and returns the size of the music
data in the mp3 file.


\section{The smallest Jukebox}

Our goal is to build a minimal ICY server. It will only stream one
tune, the idea is to get it to work and then try to extend it.  A
Server will be started, it will read a mp3 file and prepare it for
streaming. Once a media player connects it will stream the content and
then quit.

\subsection{the server}

The proxy will listen on a TCP port (8080) and wait for incoming
connection. Once a connection is accepted a request is read from the
socket. The content of the request is not important for our needs, we
will happily connect any media client and stream them the same content.

This will be the header of the server module. 

\begin{verbatim}
-module(server).

-export([start/1]).

-define(Opt,[binary, 
            {packet, 0}, 
            {reuseaddr, true}, 
            {active, true}, 
            {nodelay, true}]).

-define(Port, 8080).

-define(Header, [{'icy-name', "ID1019"}, 
                 {'icy-genre', "Groove"}, 
                 {'icy-notice1', "Our own Jukebox"}]).
\end{verbatim}

When creating the listen socket we specify the properties in a options
list. The options we will use are:

\begin{itemize}

\item {\tt binary}: more efficient since there is no point in handling
  the mp3 audio as list of integers.

\item {\tt \{packet, 0\}}: framing messages with the length of the
  message is very useful but we're dealing with the ICY protocol and not our own.

\item {\tt \{reuseaddr, true\}}: allow us to reuse the port (and we will) 

\item {\tt \{active, true\}}: the socket process will send us segments
  as they arrive, we do not have to suspend reading the socket.

\item {\tt \{nodelay, true\}}: send segments as soon as possible

\end{itemize}

The mp3 file is read by using the procedures from the mp3 module. We
are given the title as a string and the data as a huge binary. The icy
module now helps us to encode a meta data that describes the title and
also divides the data into segments.

\begin{verbatim}
start(File) ->
    spawn(fun() -> init(File) end).

init(File) ->
    .... = mp3:read_file(File),
    ... = icy:encode_meta([{title, Title}]),
    ... = icy:segments(Data),
    ... = gen_tcp:listen(?Port, ?Opt),
    server(Meta, Segments, Listen).
\end{verbatim}

We open the socket for connections and enter the state where we are
waiting for a client to connect. When a client does connect we read a
request from the socket and, if we are lucky, prepare to steam the
data segments.

\begin{verbatim}
server(Meta, Segments, Listen) ->
    ... = gen_tcp:accept(Listen),
    io:format("server: connect~n",[]),
    case read_request(...) of
        {ok, ..., _} ->
            io:format("server: received request ~p~n", [Request]),
            gen_tcp:send(..., icy:encode_response(...)),
            loop(Meta, Segments, Socket, 0);
        {error, Error} ->
            io:format("server: ~s~n", [Error])
    end.
\end{verbatim}

The reading is using our parser that could return continuations. 

\begin{verbatim}
read_request(Socket) ->
    reader(fun(More)-> ...  end, Socket).
            
reader(Cont, Socket) ->
    receive
        {tcp, Socket, More} ->
            case ... of 
                {ok, Parsed, Rest} ->
                    ...;
                {more, Fun} ->
                    ...;
                {error, Error} ->
                    ...              
            end;
        {tcp_closed, Socket} ->
            ...
    after ?TimeOut ->
           ...
    end.
\end{verbatim}

In this simple server, we simply send all the segments to the client
and when we are done we terminate. This is of course the smallest
jukebox that you have ever seen but there are enough things that can go
wrong so let's keep it simple.

\begin{verbatim}
loop(_, [], _, _) ->
    io:format("server: done ~n", []),
    ok;
loop(Meta, [{seg, Segment}|Rest], Socket, N) ->
    gen_tcp:send(Socket, Segment),
    io:format("server: sent segment ~w ~n", [N]),
    gen_tcp:send(Socket, Meta),           
    io:format("server: sent header ~w ~n", [N]),
    loop(Meta, Rest, Socket, N+1).
\end{verbatim}
 
Try to listen to your first tune delivered by the server.

\section{Your Jukebox}

So now it is time to start working, what will you jukebox look
like. You should extend the small jukebox in one or more ways so that
it can handle for example:

\begin{itemize}
\item Several concurrent clients accessing their own music stream. 

\item Let the request contain information which song that should be played.

\item Separate the sending of the audio stream from selecting the
  audio stream i.e. a handler process request segments from a music
  server process that decides which segments (and header information)
  that should be played.

\item Talk to the music server through Erlang messages to change the music.

\item Turn the jukebox into a radio station that plays songs from a pre-selected schedule.

\item Allow several client listening to the same radio stream.

\item Have a web interface to the jukebox so one can start the VLC
  client to start and play the song that is selected.
\end{itemize}

Write up a small report on problems that you encountered and how you
solved them. Describe the extension that you have done and ideas of
what could be done more.


\nnsection{Acknowledgment}

This topic of this seminar, and some of the code, is taken from the
book ``Programming Erlang: Software for a Concurrent World'' by Joe
Armstrong. The book contains more programming examples that are highly
recommended to study.


\end{document}
