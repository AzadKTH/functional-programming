\documentclass[a4paper,11pt]{article}

\usepackage[latin1]{inputenc}

\newcommand{\nnsection}[1]{
\section*{#1}
\addcontentsline{toc}{section}{#1}
}

\begin{document}

\begin{center}
\vspace{20pt}
\textbf{\large An LZW encoder}\\
\vspace{10pt}
\textbf{Johan Montelius}\\
\vspace{10pt}
\today{}
\end{center}


\nnsection{Getting started}

Lempel-Ziv-Welch (LZW) is a compression algorithm that takes advantage
of frequent occurance of sequences of characters. It will detect
sequences on the fly while doing the compression and thus create
individual codes for sequences as it goes along. The beauty of the
algorithm is that the decoder must not be told what these new codes
mean - it will learn as it does the decoding.

The encoder that we will implement will not use binary encoding
i.e. codes are fixed size and are represented by an integer. A real
implementation would start of by using, for example, a five bit code
and then increase the code length as needed. By implementing this
simpler form you will understand the principles of the algorithm and
you can easily extend it to use variable size codes.

Before you start to implement this encoder and decoder you should do
some reading on the LZW algorithm so that you have a basic
undertanding of the process. The devil is as always in the detail and
we will see how these are handled as we implement the encoder. 

\section{The table}

The encoder and decoder have to agree on an initial alphabet (and in
the general case, the code size). We will here use a very small
alphabet that consists of the smaller cap letters and the space
character. Given this we construct an initial encoder/decoder table
that is represented as a list of character sequnces and codes. 

\begin{verbatim}
-module(lzw).

-compile(export_all).

-define(Alphabet, "abcdefghijklmnopqrstuvwxyz ").

table() ->   
    N = length(?Alphabet),
    Strings = lists:map(fun(Char) -> [Char] end, ?Alphabet),
    Map = lists:zip(Strings, lists:seq(1,N,1)),
    {N+1, Map}.
\end{verbatim}

The only sequences we know of in the beginning are the sequences
consistng of single characters. We have $28$ characters in total so
our table will look like follows:

\begin{verbatim}
 {28, [{"a",1},  {"b",2},  {"c",3}, ...]}
\end{verbatim}

The number of sequences in the table is imprortant to keep track of
sicne we will add new codes as we encode our text. Have in mind that
the encoder and decoder will both know the state of the initial table.


\section{The encoder}

So let's start the encoding of a sequence of charaters. If the
sequence is empty we're done but the common case os of course if we
have at east oe charcter. We use the first character to initiate the
encoder. We pick up the encoding table, that of course holds a code
for the single character word. We then call the {\tt encode/4}
function that is given: the text, the word, the code of this word and
the coding table.

\begin{verbatim}
encode([]) ->
    [];
encode([Char|Rest]) ->
    Table = table(),
    Word = [Char],
    {found, Code} = encode_word(Word, Table),
    encode(Rest, Word, Code, Table).
\end{verbatim}

The function {\tt encode/4} is where all the actoion takes place. The
base case is simple, if there is not more characters in the text then
we're done. If we have a character in the text we add this to the
word we have read sofar and check if this extended word can be found
in the table. If we find a coding of the extended worl we're happy but we
might be even happier if we find an even longer world. This is
where we continue with the extended word and its code. 

\begin{verbatim}
encode([], _Sofar, Code, _Table) ->
    [Code];
encode([Char|Rest], Sofar, Code, Table) ->
    Extended = [Char|Sofar],
    case encode_word(Extended, Table) of
	 {found, Ext} ->
	     encode(Rest, Extended, Ext, Table);
	 {notfound, Updated} ->
            Word = [Char],
	    {found, Cw} = encode_word(Word, Table),
	    [Code | encode(Rest, Word, Cw, Updated)]
    end.
\end{verbatim}

If a code is not found for our extended word we will return a list
starting with the code of the word we had found sofar. We will then continue
the encoding 






\end{document}

