\documentclass[a4paper,11pt]{article}

% Packages for page layout
\usepackage[utf8]{inputenc}
\usepackage{fancyhdr}
\usepackage{lastpage}

% Minted for syntax highlighting
\usepackage{minted}
\usemintedstyle{tango}

% Custom paragraph indentation
\setlength{\parindent}{0em}
\setlength{\parskip}{1em}
 
% Headers/footers styling
\pagestyle{fancy}
\fancyhf{}
\renewcommand{\headrulewidth}{0pt}

% Footer
\lfoot{ID1019}
\cfoot{KTH}
\rfoot{\thepage \hspace{1pt} / \pageref{LastPage}}


% SECTIONS
%
% * Getting started
% * The table
% * The encoder


\begin{document}

% ================================================== %
% == Title == %
% ================================================== %

\title{
    \textbf{A LZW Encoder}\\
    \large{Programming II - Elixir Version}
}
\author{Johan Montelius}
\date{Spring Term 2018}
\maketitle
\thispagestyle{fancy}


% ================================================== %
% == Getting started  == %
% ================================================== %

\section*{Getting started}

Lempel-Ziv-Welch (LZW) is a compression algorithm that takes advantage
of frequent occurrence of sequences of characters. It will detect
sequences on the fly while doing the compression and thus create
individual codes for sequences as it goes along. The beauty of the
algorithm is that the decoder must not be told what these new codes
mean - it will learn as it does the decoding.

The encoder that we will implement will not use binary encoding
i.e. codes are fixed size and are represented by an integer. A real
implementation would start off by using, for example, a five-bit code
and then increase the code length as needed. By implementing this
simpler form you will understand the principles of the algorithm and
you can easily extend it to use variable size codes.

Before you start to implement this encoder and decoder you should do
some reading on the LZW algorithm so that you have a basic
understanding of the process. The devil is as always in the detail and
we will see how these are handled as we implement the encoder.


% ================================================== %
% == The table  == %
% ================================================== %

\section{The table}

The encoder and decoder have to agree on an initial alphabet (and in
the general case, the code size). We will here use a very small
alphabet that consists of the smaller cap letters and the space
character. Given this we construct an initial encoder/decoder table
that is represented as a list of character sequences and codes.

\begin{minted}{elixir}
defmodule LZW do

  @alphabet 'abcdefghijklmnopqrstuvwxyz '
  
  def table do
    n = length(@alphabet)
    numbers = Enum.to_list(1..n)
    map = List.zip([@alphabet, numbers])
    {n + 1, map}
  end
  
end
\end{minted}

The only sequences we know of in the beginning are the sequences
consisting of single characters. We have $28$ characters in total so
our table will look like follows:

\begin{minted}{elixir}
{28, [{97, 1}, {98, 2}, {99, 3}, ...]}
\end{minted}

The number of sequences in the table is important to keep track of
since we will add new codes as we encode our text. Have in mind that
the encoder and decoder will both know the state of the initial table.


% ================================================== %
% == The encoder  == %
% ================================================== %

\section{The encoder}

So let's start the encoding of a sequence of characters. If the
sequence is empty we're done but the common case is of course if we
have at east one character. We use the first character to initiate the
encoder. We pick up the encoding table, that of course holds a code
for the single character word. We then call the {\tt encode/4}
function that is given: the text, the word, the code of this word and
the coding table.

\begin{minted}{elixir}
def encode([]), do: []

def encode([word | rest]) do
  table = table()
  {:found, code} = encode_word(word, table)
  encode(rest, word, code, table)
end
\end{minted}

The function {\tt encode/4} is where all the action takes place. The
base case is simple, if there is not more characters in the text then
we're done. If we have another character in the text we add this to the
word we have read so far and check if this extended word can be found
in the table. If we find a coding of the extended word we're happy but we
might be even happier if we find an even longer world. This is
where we continue with the extended word and its code.

\begin{minted}{elixir}
def encode([], _sofar, code, _table), do: [code]

def encode([word | rest], sofar, code, table) do
  extended = [word | sofar]
  case encode_word(extended, table) do
    {:found, ext} ->
      encode(rest, extended, ext, table);
    {:notfound, updated} ->
      {:found, cd} = encode_word(word, table)
      [code | encode(rest, [word], cd, updated)]
  end
end
\end{minted}

If a code is not found for our extended word we will return a list
starting with the code of the word we had found so far. We will then continue the encoding.

\end{document}
