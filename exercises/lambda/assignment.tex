\documentclass[a4paper,11pt]{article}

\usepackage[latin1]{inputenc}

\newcommand{\nnsection}[1]{
\section*{#1}
\addcontentsline{toc}{section}{#1}
}

\usepackage{syntax}


\begin{document}

\begin{center}
\vspace{20pt}
\textbf{\large Elixir, functional programming and\\ the Lambda calculus.}\\
\vspace{10pt}
\textbf{Johan Montelius}\\
\vspace{10pt}
\today{}
\end{center}

\newcommand{\lamc}[0]{$\lambda$-calculus}
\newcommand{\lama}[0]{$\lambda$-abstraction}
\newcommand{\lame}[0]{$\lambda$-expression}
\newcommand{\alphac}[0]{$\alpha$-conversion}
\newcommand{\betar}[0]{$\beta$-reduction}
\newcommand{\etac}[0]{$\eta$-conversion}

\nnsection{Introduction}

In this tutorial you're going to explore lambda calculus and how it
relates to functional programming. We're going to look at some
examples using Elixir to see how a functional programming language can
be expresses in lambda calculus but before we begin, you need a bit of
historical background.

Lambda calculus, or \lamc, was introduced by Alonzo Church in 1932. It
was a formal description of mathematics and used {\em function abstractions}
and {\em function applications} as the basis. The calculus was used in
mathematics to study computability and can be shown to be Turing
complete i.e. anything computable can be computed using \lamc.


The beauty of \lamc\ is that it is so simple, it consist of very few
constructs and rules, yet it is as powerful as the most powerful
programming language. No one in their right mind would program
anything in \lamc, but it has served as the inspiration of a whole
family of languages.

In the dawn of computers the only programming language was the language of
the hardware i.e. assembler. In the late fifties computers were
however powerful enough to allow programs, written in high-level
languages, to be compiled to machine code. One of the first high-level
languages that were defined was Lisp. It was developed by John
McCarthy in 1958 and was directly based on \lamc. It introduced many
features of programming languages that we now take for granted.

Lisp has been followed by a number of functional
programming languages: {\em ML}, {\em Scheme}, {\em Clojure}, {\em
  F\#}, {\em Haskell}, {\em Scala} and many more. The functional
programming paradigm has also influenced traditional imperative
languages so we now have so called {\em lambda expressions} even in C++.

Since the \lamc\ has been so influential in the development of
programming languages it fun to know what it is all about.

\section{\lamc}

We will start by explaining the rules of \lamc\ by using the basic
arithmetic operations as an example. This is a bit of cheating since we
then don't really explain how the basic arithmetic operations are
defined but it's a good start in understanding \lamc.

\subsection{\lama}

You all understand what I mean if I write a mathematical expression
like the following:$$f(x) = 2x + 5$$

We have defined a function $f$ that takes one argument $x$ and the
function is defined as $2x + 5$. You also take it for granted that
$f(2)$ is equal to $9$ and the \lamc\ is not much more complicated
than this. What \lamc\ does, is that it formally describes the rules
that we have to apply in order to determine that $f(2)$ evaluates to
$9$.

The first thing we have to get used to is that functions do not have
names. Any sane programming language will of course add names to
functions but if we want to describe the bare minimum we do away with
anything that will complicate things. If we have no names we can of
course not write $f(2)$ but you will see that we can express the same
thing without names.

The function above, would in \lamc\ be written as
follows:$$\lambda x \rightarrow 2x + 5$$The traditional way of writing
this is $\lambda x.2x+5$ but we will use an arrow instead since it
will then map very natural to the Elixir syntax. 

The expression $\lambda x \rightarrow 2x + 5$ is a called a {\em
  \lama}, we're abstracting a parameter $x$ from an {\em
  expression}. As we in regular arithmetic can write $f(2)$ we can in
\lamc\ {\em apply} an abstraction to an expression. We
write: $$(\lambda x \rightarrow 2x + 5) 2$$

We have here used parentheses to make it clear that the abstraction is
one expression and we apply it to the the expression $2$. Parentheses
can often be omitted (application is left associative and expressions
extends as far right as possible) but we keep them here to make it
clear what we mean.

\subsection{\alphac}

When we reason about \lamc\ we will often refer to the {\em free
  variables} of an expression. Free variables are variables that are
not {\em bound} by a \lama. In the expression $2x + y$, $x$ ad $y$ are
free variables but in the expression $\lambda x \rightarrow 2x + y$
only $y$ is free. The variable $x$ is then said to be {\em bound}.

A \lama\ can be rewritten by what is called {\em \alphac}. We will then
simply rename a variable that is bound in an abstraction, but the
meaning will remain the same. The abstraction
$\lambda x \rightarrow x +2$ is of course identical to
$\lambda z \rightarrow z +2$ - the name of the variable does not
matter. However, when we do this conversion we have to be careful if
we replace $x$ for $z$ in the abstraction below, the result is not the
same (try):$$\lambda x \rightarrow x + z$$

The rules for \alphac\ are not trivial since we can easily do the
wrong thing. For example, what would it mean to replace $x$ for $z$ in
the abstraction below? $$\lambda x \rightarrow (\lambda z \rightarrow x + z)$$

In general we are not allowed to do a conversion that binds an
otherwise free variable. We can define this in a more formal way but
this is just an overview of \lamc.

\subsection{\betar}

A \betar\ is what we do when we apply a \lama\ to an argument. We will
{\em substitute} the variable of the abstraction for the parameter. We
write substitution using the following notation $$(2x + 5)[x/3]$$ We
can write down some formal rules for what it means to do substitution
but it is all quite simple. The only thing we have to be careful
about is when we want to substitute one variable by another. This
could of course lead to very strange situations. 

If we have the abstraction
$\lambda x \rightarrow (\lambda y \rightarrow y + x)$ we can
substitute $[x/3]$ and receive $\lambda y \rightarrow y + 3$ but what
happens if we substitute $[x/y]$? This would give us
$\lambda y \rightarrow y + y$ which is probably not what we mean. This
is were we use \alphac\ to first transform the expression to avoid
name clashes. If we first transform the abstraction to
$\lambda x \rightarrow (\lambda z \rightarrow z + x)$ we have no
problem substituting $[x/y]$.

\subsection{\etac}

There is only one more rule that we need and this is \etac\
(eta-conversion). This rules states that if we have an abstraction,
$\lambda x \rightarrow f x$, where $f$ is an expression in which $x$
does not occur free, then we can convert this to simply $f$. This is a
quite simple rule that allows us to reduce something that is not
needed. If we have the abstraction
$\lambda x \rightarrow (\lambda y \rightarrow y + 3) x$ this is of
course the same thing as $\lambda y \rightarrow y + 3$.  Applying the
first abstraction to $5$ will result in the expression
$(\lambda y \rightarrow y + 3) 5$ so we might as well apply
$\lambda y \rightarrow y + 3$ to $5$ in the first place.

\subsection{that's all}

The three rules: \alphac, \betar\ and \etac, are the only rules we
need to define \lamc. We can of course spend some more pages on the
formal description of the rules and the syntax, but the key point is
that it's all very simple. You might rightly ask two questions; if
it's so simple can we actually use it for anything and, what about
these arithmetic operations, where are those defined? 

The answer to the first question is that the language that we have is
as powerful, in a theoretical sense, as any programming language that
you know or will ever learn. The answer to the second question is -
magic, we don't really need numbers nor primitive arithmetic
operations; we only used them to introduce the rules of \lamc (more on
this later).

\section{Functional programming}

To see how functional programming relates to \lamc\ we will show how
we can express an Elixir program in terms of \lame s. We will see how
different functional programming languages have made different choices
but that they all have their roots in \lamc.

The connection between Elixir and \lamc\ is easy to see if we show how
Elixir writes anonymous functions. The \lama\
$\lambda x \rightarrow x + 3$ is in Elixir written:

\vspace{10pt}
\begin{center}{\tt fn(X) ->  X  + 3 end}
\end{center}
\vspace{10pt}

The $\lambda$ notation is replaced by {\tt fn}, the parameters are
enclosed by parenthesis and the expression is terminated by {\tt
  end}. When we apply a function to an argument the only difference
from the \lamc\ is that we enclose the arguments in parenthesis.

\vspace{10pt}
\begin{center}{\tt fn(X) ->  X  + 3 end.(5)}
\end{center}
\vspace{10pt}

If you have not done so all ready, start up a Elixir shell and
experiment with functional expressions. Try this:

\begin{verbatim}
> add3 = fn(x) -> x + 3 end
 :
> add3.(5)
 :
\end{verbatim}

There is however more to it than just different syntax. The most
important difference is the order in which things are evaluated.


\subsection{order of evaluation}

The \lamc\ does not define in what order the rules should be applied. The
interesting thing is that it, with some exceptions, does not matter. If we
have the expression
$$(\lambda x \rightarrow (\lambda y \rightarrow x + y) 2) 3$$ 
it does not matter if we do the innermost reduction first
$$(\lambda x \rightarrow x + 2) 3$$ 
or the outermost 
$$ (\lambda y \rightarrow 3 + y) 2$$ 
in the end the result will be the same, $5$.

Is this always the case you might wonder and unfortunately it is not.
Sometimes the order of evaluation matters. Look a the mysterious
looking expression below:
$$(\lambda r \rightarrow r r)(\lambda r \rightarrow r r)$$ 
If we apply the leftmost abstraction to the argument we will duplicate
this and apply it to itself. The result is
$$(\lambda r \rightarrow r r)(\lambda r \rightarrow r r)$$ 
but this has led us nowhere, we're stuck in an infinite loop. Make a
note of this: it is possible that an evaluation will never return an answer.

Now take a look at the abstraction below:
$$\lambda x \rightarrow (\lambda y \rightarrow x)$$ 
This is an abstraction that when applied to an argument would give us
an abstraction. This abstraction takes an argument but simply returns
the original argument. If you carry out the required \betar\ for the
expression below, I think the result will be $3$. It does not really
matter what we have instead of $2$, the result will always be $3$.
$$(\lambda x \rightarrow (\lambda y \rightarrow x) 2) 3$$ 


The problem is if we instead of $2$ write the abstraction that will
only result in a looping computation.
$$(\lambda x \rightarrow (\lambda y \rightarrow x) ((\lambda r \rightarrow r r)(\lambda r \rightarrow r r)) ) 3$$ 
If we're smart we will of course apply $(\lambda y \rightarrow x)$
first and get away with the looping expression and then apply the
abstraction $(\lambda x \rightarrow x)$ to $3$. If we're not so smart
we will spend the rest of our lives evaluating the loop.

Any programming language would have to define the order of evaluation
so that a programmer could avoid infinite loops and be able to
estimate the run-time complexity of a program. In Elixir, as in most
languages, the rule is that the that the arguments are evaluated
first, before the function is applied. This is called {\em eager
  evaluation}, {\em applicative order} or {\em call-by-value}. Some
languages, most noticeably Haskell, take the opposite approach and
applies the function first and evaluate the arguments only if
needed. This is called {\em lazy evaluation}, {\em normal order} or
{\em call-by-name}.

There are pros and cons with either strategy so one should not take
one for granted. This small tutorial on the subject is too short to
look at the different strategies. The important thing is that
different strategies exist and that they are equal modulo infinite
computations (and exceptions).

\begin{quotation} {\em If one strategy produces a result then any
    strategy will, if it terminates, produce the same result. }
\end{quotation}

This property is something that a compiler or run-time system can make
use of. Without fear of doing the {\em wrong} thing, it can choose to
modify the evaluation order or do evaluation for example in
parallel. The behavior must of course be predictable in time and
space requirements. Changes made by the system should only improve
things i.e. never go into an infinite loop if the evaluation order of
the programming language would not go into a loop.

\subsection{more than one argument}

One limitation with the \lamc\ syntax is that it only allows one
parameter to a function. This might seem like a serious deficiency but
computation-wise it does not make a difference. If we extended the
syntax of the language we could of course write something like this:
$$\lambda x y \rightarrow x + y$$

This is a harmless extension to the language since we can always
rewrite it in terms of \lama\ that only take one argument. The above
expression would be written:
$$\lambda x \rightarrow (\lambda y \rightarrow x + y )$$

Applying one argument after the other is rather complicated and
therefor any functional programming language have a syntax that allows
more arguments. The important thing to note is, that we do not have to
change or add any rules to the calculus. It is all just syntactic sugar
that is it makes things easier to read and write.

While using several arguments to a function make life easier the
we sometimes want to do the opposite. We want to
turn a function of several parameters to a sequence of functions of
one parameter each. This is so important that it has been given a
name, {\em currying}, after the mathematician Haskell B. Curry (who also
gave name to the language Haskell).

In languages where currying is provided by the compiler, one can define
a function with for example two arguments and then apply it to its
first argument to get a specialized function in return. This is not
possible in Elixir, but if it was, one would be able to write something
like the following:

\vspace{10pt}
\begin{center}{\tt f = fn(x, y) ->  x  + y end; f3 = f.(3); f3.(5)}
\end{center}
\vspace{10pt}

Here, {\tt F} is function that takes two arguments but {\tt F3} is a
specialization that will add {\tt 3} when it is applied to a second
argument. If we want to do something like this in Elixir we would have
to do it by hand. Try this in an Elixir shell:

\vspace{10pt}
\begin{center}{\tt f = fn(x) -> fn(y) ->  x  + y end end; f3 = f.(3); f3.(5)}
\end{center}
\vspace{10pt}

\subsection{let-expressions}   

If you have done some Elixir programming you have of course seen how
an Elixir function consists of a {\em head} and a {\em body}, where the
body is a sequence of expressions. We can for example write the
following function in Elixir.

\vspace{10pt}
\begin{center}{\tt fn(x) ->  y = x+4; y + y end}
\end{center}
\vspace{10pt}

How is this translated into \lamc? There is nothing called a sequence
in \lamc\ so we have to rewrite it in terms of regular language
constructs. We rewrite the abstraction as follows:

\vspace{10pt}
\begin{center}{\tt fn(x) -> (fn(y) -> y + y end).(x+4) end}
\end{center}
\vspace{10pt}

We now have an function of one argument that will apply the
function {\tt fn(y) -> y + y end} to the argument {\tt x+4}.  We
now have a form that correspond to the regular \lamc\ syntax.

$$\lambda x \rightarrow (\lambda y \rightarrow y + y) x+4$$

The shorter form is often referred to as a {\em let-expression} and is
read as {\em let the variable \ldots hold the value \ldots in the
  expression \ldots}. We find this construct in many functional
programming languages, this is for example how it is written in Lisp:
\begin{center}{\tt  (lambda (x) (let (y (+ x 4))  (+ y y) ))}
\end{center}
and in Haskell we would write it as follows:
\begin{center}{\tt  (\textbackslash x  -> let  y = x + 4  in  y + y)}
\end{center}

There are limitations on what we allow in let-expressions. One
thing we cannot write in Elixir is the following:

\begin{center}{\tt fn(x) -> l = [x|l];  l end}
\end{center}

If we would try to turn this into a form of a \lame\ we will have a
problem with the infinite list {\tt l}. In some functional
programming languages however, this is perfectly fine. Both Scheme and
Haskell allow let-expressions to have recursive definitions. In Scheme
we can write:

\begin{center}{\tt  (lambda (x) (letrec ((l (cons x (delay l)))) l))}
\end{center}

We then have to explicitly state that we should not evaluate the
variable $l$ when declaring $l$. We can later force the evaluation if
we want to go through the list of infinite numbers. In Haskell, it is
so common to use infinite structures that it is provided by default. 

\begin{center}{\tt  (\textbackslash x  -> let  l = x:l  in  l)}
\end{center}

Haskell is as we mentioned a language that uses lazy-evaluation so it
is more natural to work with infinite structures. Working with
infinite structures is quite fun and you can write very nice program
(that are mind-boggling) but in Elixir it is not allowed (or rather we
have to construct them manually).

\subsection{recursion}

One thing that we have avoided so-far is how to express recursion. In
Elixir, or any programming language, defining a recursive function is
as simple as writing {\tt append}.

\begin{verbatim}
def append(a, n)  do
  case a do
    [] -> b
    [h|t] -> [h|append(t,b)]
  end    
\end{verbatim}

How do we define a recursive function if we don't have named
functions? We could do it in Scheme since we there have the {\tt
  letrec} construct. We define a local variable, {\tt app}, as a
lambda expression and then use this local variable in a recursive
call.

\vspace{10pt}

\begin{verbatim}
 (lambda (a b) 
      (letrec ((app (lambda (x y) 
                           (if (equal? x '()) 
                                   y 
                                   (cons (car x) (app (cdr x) y))
                           )))) 
               (app a b))
\end{verbatim}
\vspace{10pt}

If we can not do this in plain \lamc\ then we have a problem. We need
to be able to express recursive functions or programs would have to be
infinite in size which is not very practical. Can we do a trick and
define recursive functions without having to extend the \lamc? Have a
look at this:

\begin{verbatim}
def append(a, b) ->
  app = fn(x, y, f) -> 
          case x of 
            [] -> Y
            [h|t] -> [h | f.(t,y,f) ]
          end
        end,
   app.(a, b, app)
end
\end{verbatim}

Hmm, we're defining a local function, {\tt app}, that takes three
arguments, two lists and a mysterious third argument {\tt f}. We then
hope that {\tt f} will save us when it is time for the recursive
call. Will this work? Well, look at how we are using {\tt app}, when
we call it with the arguments {\tt a} and {\tt b} we pass {\tt app}
itself as the third argument. We're not violating the rules of \lamc\
and yet we have managed to express a recursive function.

There is a general way to do this and the trick is to use something
called the {\em Y-combinator}. It can be used to transform any
recursive definition to a non recursive definition.

$$\lambda f \rightarrow (\lambda x \rightarrow f(x x)) (\lambda x \rightarrow f (x x))$$

If you want to know more about how this works you should learn about
{\em fix points} and how they can be used. Exactly how this is done is outside
the scope of this tutorial, but now you have seen the Y-combinator.

To see if you can implement a recursive function without using a
recursive definition you can try to implement the Fibonacci
function. Take a look at the implementation of {\tt append/2} and try
to do something similar.

\subsection{named functions}

Even though it is possible to define recursive functions without
having to use names it is a whole lot more practical to use names.
All functional programming language introduce named functions and
Elixir is not an exception.

Even though we have named functions the ability to use nameless
functions and pass them as arguments to other functions is a very
powerful technique. Since this is the core of \lamc\ we know exactly
what it means and there is nothing strange about it. It is of course
strange if you approach functional programming from an imperative
programming language but now you know the basics of \lamc\ and will
after some practice take it for natural.

\section{Church numerals}

In the beginning we noted that we were cheating a bit by using common
arithmetic operations. We said that we did this in order to more
easily introduce \lamc\ but that it was not strictly needed. Since you
of course wonder how we could do without basic data structures as
number and simple arithmetic operations as addition we will now
explain the mystery of {\em Church numerals}.

If we only have \lame s we could not express very much but we could
express something. We could express the following interesting
function, a function that ignores its argument and simply returns a
function that will return its argument. 

$$\lambda f \rightarrow \lambda x \rightarrow x$$

Another function is a function that returns a function that applies
the argument to the argument of that function:

$$\lambda f \rightarrow \lambda x \rightarrow f x$$

We could then describe functions that applies the argument twice
or three times and this is the trick used by Church numerals. We
represent the natural numbers by functions on the above form. The number $0$
does not use the argument at all while $4$ applies the argument four
times. This is how we would write the number $4$:

$$\lambda f \rightarrow \lambda x \rightarrow f (f (f (f x)))$$

If this is how we represent numbers the question is if we can
use them and for example express addition. Addition is a
function that takes two arguments, both encoded as above, and returns
a function that represents the addition of the two arguments. How
about this (we allow the \lama s to take two arguments):

$$\lambda a b \rightarrow \lambda f x \rightarrow a f (b f x)$$

Since this becomes quite messy to write, and since you should learn
some Elixir programming in the course, we could try to implement this
in Elixir. We first write two functions, one that will generate a
Church numeral from an integer and one that will turn a Church numeral
into an integer. Create a file {\tt church.ex} and get your Elixir
system ready, you will have to do a lot of experimenting to understand
what is going on.

\begin{verbatim}
defmodule Church do

  def to_church(0) do
    fn(_, y) -> y end
  end
  def to_church(n) do
    fn(f, x) -> f.( (to_church(n-1)).(f, x) ) end
  end

  def to_integer(church) do
    church.(fn(x) -> 1 + x end , 0)
  end
\end{verbatim}

Note how simple it is to implement the function that turns a Church
numeral into an integer. We simply ask that numeral to apply the
function {\tt fn(x) -> 1 + x end} as many times as it is supposed to
do on the number {\tt 0}. Compile the file and try this in the Elixir shell:

\begin{verbatim}
> four = Church.to_church(4).
 :
> Church.to_integer(four).
 :
\end{verbatim}

Now we can define the function {\tt succ(n)} that will take a numeral
and return a numeral that is the {\em successor} i.e. one more. We
can do this simply by returning a function that takes two arguments, a
function {\tt f} and some value {\tt x}, and uses {\tt n} to apply {\tt f}
{\tt n} times before applying {\tt f} again on the result. 

\begin{verbatim}
  def succ(n) do
    fn(f, x) -> f.( n.(f, x) ) end
  end
\end{verbatim}

Compile and see what the successor of {\tt four} is; does it work?

We can write addition as outlined in the \lame\ above and the product
is almost as simple. Do some experiment with these functions to see
that they actually do what they are supposed to do.

\begin{verbatim}
  def add(n, m) do
    fn(f, x) -> n.(f, m.(f,x)) end
  end
\end{verbatim}

\begin{verbatim}
  def mul(n, m) do
    fn(f, x) -> n.( fn(y) -> m.(f, y) end, x) end
  end
\end{verbatim}

If you think this was easy you can try to figure out how to write the
{\em predecessor} function. The predecessor should of course do the
opposite of the successor function and the only question is what the
predecessor of zero should be. Since we do not have any representation
of negative numbers we simply state that the predecessor of zero is
zero.

If you give this challenge some thought it will turn out to be a quite
tricky problem. If you don't Google it, or continue to read this
assignment, it will probably take you three years to figure out how to
do it. Even if you read the following and realize that it works, it is
very likely that you will not be able to recreate it tomorrow. OK,
you're warned - here we go.

When defining {\tt pred(n)} we must of course make use of {\tt n}
somehow. The trick is to apply {\tt n} to two functions where the
first function is a function that returns a function. When we are done
with {\tt n} we will take the result, which is a function, and apply it
to the identity function; I know this does not make sense. 

\begin{verbatim}
  def pred(n) do
    fn(f, x) ->
      ( n.(                                    # n is a Church numeral
         fn(g) -> fn(h) -> h.(g.(f)) end end,  #  apply this function n times 
         fn(_) -> x end)                       #  to this function 
      ).(fn(u) -> u end)                       # apply it to thee identity function
    end
  end
\end{verbatim}

Assume we call {\tt pred(four)} where {\tt four} is the Church numeral
for $4$. Then we will return a function, that as expected takes two
arguments, a function {\tt f} and and argument {\tt x}. To see how
this function works let's apply it to {\tt i} and {\tt 0}.

First we will call {\tt four}, with a strange looking function and a
function that ignores its argument. Let's apply this functions once,
what do we get?

\begin{verbatim}
    fn(h) -> h.( (fn(_) -> 0 end).(i)) end 
\end{verbatim}

We have taken the function that ignores its first argument and
substituted it for $g$. We now apply this function to {\tt f} and get
the function.

\begin{verbatim}
    fn(h) -> h.(0) end
\end{verbatim}

So after one recursion we have replaced the function that ignores its
argument with a function that applies it once.  We have three more
recursions before we're ready. The next recursion will give us
something that looks like follows:

\begin{verbatim}
    fn(h) -> h.( (fn(h') -> h'.(0) end).(i)) end 
\end{verbatim}

This can of course be written:
  
\begin{verbatim}
    fn(h) -> h.( i.(0) ) end     
\end{verbatim}
 
Hmmm, let's give it another shot:

\begin{verbatim}
   fun(H) -> H( fun(H1) -> H1( I(0) ) end end (I)) end 
\end{verbatim}
or
\begin{verbatim}  
   fn(h) -> h.(i.(i.(0)) ) end 

\end{verbatim}

OK, a last time:
\begin{verbatim}
    fn(h) -> h( fn(h') -> h'.(i.(i.(0))) end).(i) end 
\end{verbatim}
or
\begin{verbatim}  
    fn(h) -> h.(i.(i.(i.(0)))) end 
\end{verbatim}

And now we take this function and apply it to the identity function
{\tt fn(u) -> u end}. The result is:

\begin{verbatim}
    i(i(i(0)))
\end{verbatim}

This is of course exactly what we would like to see from {\tt
  pred(four).(i,0)}. You should give it a try
and see that it works. We try the following:

\begin{verbatim}
 > three = Church.pred(four).
 :
 > Church.to_integer(three).
 :
\end{verbatim}

I you think this was complicated you're right! Church never figured
out how to express the predecessor function. It one of his students,
Stephen Kleene, that four years later showed his professor that the
\lamc\ could express something as simple as the predecessor function. 

Now that we have the predecessor function we can implement a function
{\tt minus/2} that does subtraction. The implementation is simpler
than what you might first think. Here are some hints to get you starting:


\begin{itemize}
\item Minus: use the function {\tt pred/1} defined before and
  implement {\tt minus/2}. Think like this: $m - m$ is simple, if I
  have a function that subtracts one (hint {\tt pred/1}) I will use
    $n$ to apply it $n$ times on $m$. 
\end{itemize}

There are more exercises in Church encoding, how to express Boolean
values, conditional expressions, equality etc. The important thing is
not exactly how this is done but that it can be done. It proves that
\lamc\ is {\em Turing complete} i.e. any efficiently computable
function can be computed using \lamc.

To train your skills in Elixir programming and get some
more insight into Church numerals you can try the following challenges:

\begin{itemize}

\item Boolean value: represent {\em true} as a function that takes two
  arguments and returns the first, and {\em false} as a function that
  returns the second.  Implement the Elixir functions {\tt church_true/0} and
  {\tt church_false/0} that return these functions.
  
\item Boolean check: implement a function, {\tt to_boolean/2},  that takes a Church Boolean
  and returns {\tt true} if it is true and {\tt false} if it is false
  (the solution is very simple).
  
\item Boolean and, or, not: implement the basic Boolean operators. You have to
  call them something like {\tt church_and/2} etc since {\tt and},
  {\tt or} and {\tt not} are the predefined Boolean operators.

\item is_zero : implement a function that returns true if the argument
  is zero, otherwise false. This is tricky but think like this: if $n$
  is a numeral, you can apply it to a function and an argument. If $n$ is
  zero it will not apply the function and will thus return the
  argument (which should then be?). If $n$ is one it will apply the
  function once to the argument but return false (so the function should be?).

\item Les-than-or-equal: implement the function {\tt leq/2} that
  returns true or false depending on if the first argument is less
  than or equal to the second argument. Remember that $3-5$ is zero so
  $n$ is less than or equal to $m$ if $n - m$ is zero.
\end{itemize}

If you have done the functions above we will take a look at one that
looks very simple but that puts its finger on a very interesting
property - why eager evaluation might not be the silver bullet we take
it for granted to be. Let's begin with a simple solution that we think
will work.


\begin{itemize}
\item If then else: implement a function {\tt if_then_else/3} that
  does what it is called. Think like this - a Church Boolean takes two
  arguments and returns either the first or second depending on if it
  is true or false. Our first argument to {\tt if_then_else/3} is the
  Church Boolean that we should use to either return {\tt Then} or {\tt Else}. 
\end{itemize}

If you implemented it as simple as possible it should be a one line
function, I told you it was simple. Now let's use {\tt if_then_else/3}
to implement subtraction is a new way. Fill in the blanks in this code
(let's call it {\tt manos/2}) and give it a try:

\begin{verbatim}
  def manos(m, n) do
    if_then_else(is_zero(n), m, manos(m, pred(n)))
  end
\end{verbatim}

Could not be easier, give it a try (define the Church numerals {\tt
  four} and {\tt three} first):

\begin{verbatim}
 > Church.manus(four, three).
   :
\end{verbatim}

Hmm, something is not quite right - what is going on and why? In what
order do we evaluate the function?  Can we rewrite the {\tt
  if_then_else/3} in any way to avoid the problem? Do we actually want
to be slaves under the eager evaluation strategy? How would this work
in Haskell?  Maybe we need a construct in Elixir that avoids the eager
evaluation order? Should we have special function that does not
evaluate all arguments, a special notation that the evaluation should
be delayed or a language construct?

As you see, a simple thing as {\em if-then-else} is quite tricky. You
now also know why Elixir and most functional programming languages have
a language construct that implements the if-then-else behavior that we
are looking for. In Elixir we have {\em case expressions} and {\em if
  expressions} that gives us the functionality that we would not
otherwise get.

\section{Summary}

If you have implemented the Church numerals I think that you have a
better understanding of what the \lamc\ is. I also hope that you're a
little bit better in working with functions as arguments and return
values. This is something that you will not use very much in the
beginning but once you get use to it you will realize how powerful it
is.

You do not need to understand all the details of \lamc , and no one
will probably never ask you to implement the Church numerals again,
but having a basic understanding of \lamc\ will give you a better
understanding of how different functional programming languages are
related. The syntax may differ and they have adopted different
strategies for order of evaluation but they are all rooted in the same
paradigm. Learn the paradigm and you will easily learn any functional
programming language.

\end{document}
