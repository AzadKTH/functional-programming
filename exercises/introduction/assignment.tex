\documentclass[a4paper,11pt]{article}

% Packages for page layout
\usepackage[utf8]{inputenc}
\usepackage[T1]{fontenc}
\usepackage{fancyhdr}
\usepackage{lastpage}

% Minted for syntax highlighting
\usepackage{minted}
\usemintedstyle{tango}

% Custom paragraph indentation
\setlength{\parindent}{0em}
\setlength{\parskip}{1em}
  
% Headers/footers styling
\pagestyle{fancy}
\fancyhf{}
\renewcommand{\headrulewidth}{0pt}

% Footer
\lfoot{ID1019}
\cfoot{KTH}
\rfoot{\thepage \hspace{1pt} / \pageref{LastPage}}


% SECTIONS
%
% * Getting started
% * Simple arithmetic
% * A first program
% * Recursive definitions
% * List operations
% * Reverse
% * More challenges
% * Fibonacci


\begin{document}

% ================================================== %
% == Title == %
% ================================================== %

\title{
    \textbf{Elixir: getting started}\\
    \large{Programming II - Elixir Version}
}
\author{Johan Montelius}
\date{Spring Term 2018}
\maketitle
\thispagestyle{fancy}


% ================================================== %
% == Getting started == %
% ================================================== %

\section*{Getting started}

We assume that you have Elixir up and running and an editor to write
your first Elixir programs. You don't need a full IDE, rather the
less you have to think about the better.


In this tutorial we will write input to the Elixir shell as follow:

\begin{minted}{elixir}
> x + 2
\end{minted}

You will actually see a number {\tt iex(1)>}, but we will simply
show a single {\tt >}.  


% ================================================== %
% == Simple arithmetic == %
% ================================================== %

\section{Simple arithmetic}

Open up a Elixir shell and try some simple arithmetic. Try these examples:

\begin{minted}{elixir}
> 6 + 2

> 6 / 2

> div(7,2)

> rem(7,2)

> rem(-1,5)
\end{minted}

Some simple comparisons:

\begin{minted}{elixir}
> 3 == 3

> 3 != 4

> 4 < 7
\end{minted}

Try this:

\begin{minted}{elixir}
> h()
\end{minted}


% ================================================== %
% == A first program == %
% ================================================== %

\section{A first program}

Open up a file {\tt test.ex} and create a module called {\tt Test}. In
this module we define a function called {\tt double/1} that takes one
argument and returns the double of that argument.

\begin{minted}{elixir}
defmodule Test do

  # Compute the double of a number.
  def double(n) do
    ...
  end
  
end
\end{minted}

In a regular shell you can now compile this program using the stand alone compiler:

\begin{minted}{bash}
$ elixirc test.ex
\end{minted}

You can also compile and run the program from within the Elixir shell:

\begin{minted}{elixir}
> c("test.ex")

> Test.double(4)
\end{minted}

Now in the same module define the following functions:

\begin{itemize}
\item a function that converts from Fahrenheit to Celsius (the
  function is as follows: $C = (F-32)/1.8$)

\item a function that calculates the area of a rectangle give the
  length of the two sides

\item a function that calculates the area of a square, using the
  previous function

\item a function that calculates the area of a circle given the radius
\end{itemize}

Sooner or later you will have to think about which programming
environment that you are going to use. We will not write very large
programs in the course, but you will need to be able to quickly
re-compile programs and shift between the editor and the Elixir
shell. There are many environments to choose from and you are of course
free to use whichever you find best.


% ================================================== %
% == Recursive definitions  == %
% ================================================== %

\section{Recursive definitions}

Assume that all we have is addition and subtraction but need to define
multiplications. How would you do? You will have to use recursion and
you solve it by first describing the multiplication functions by words.

{\em The product of m and n is: $0$ if $m$ is equal to $0$, otherwise the
  product is $n$ plus the product of $m-1$ and $n$.}

Once you have written down the definition, the coding is simple.

\begin{minted}{elixir}
def product(m, n) do
  if m == 0 do
    ...
  else
    ...
  end
end
\end{minted}

There are alternative ways of writing this, we could use a {\em case}
  expression or the {\em cond-do} expression. These techniques are often
  preferred over {\em if-else} in Elixir.
  
\begin{minted}{elixir}
def product_case(m, n) do
  case m do
    0 ->
      ...
    _ ->
      ...
  end
end

def product_cond(m, n) do
  cond do
    m == 0 ->
      ...
    true ->
      ...
  end
end
\end{minted}

We could also have used the a style that sometimes is very handy. Here
we break the definition up into two {\em clauses} that are tried one
after the other.

\begin{minted}{elixir}
def product_clauses(0, _) do 0 end
def product_clauses(m, n) do
  product_clauses(..., ...) + ...
end
\end{minted}

This should be read: if we call product, and the first argument
matches {\em 0}, then the result is {\em 0}. If we can not use the
first clause then we try the second.

Sometimes the code becomes easier to understand, especially if we have
many conditions that should be tested. Remember though that the
clauses of a function need to be after each other. You can not spread
the clauses around in a program.


Define a function, {\tt exp/2}, that computes the exponentiation,
$x^n$. Use only the addition and subtraction and the function {\tt
  product/2}, that you defined.
  
\begin{minted}{elixir}
def exp(x, n) do
  case ... do
    ...
  end
end
\end{minted}

Use the built-in arithmetic functions {\em rem}, {\em div} and
multiplication {\em $*$} to implement a much faster exponentiation using
the following definition:

\begin{itemize}
  \item $x$ raised to 1 is $x$
  \item $x$ raised to $n$, if $n$ is even, is $x$ raised to $n/2$ multiplied by itself
  \item $x$ raised to $n$, if $n$ is odd, is $x$ raised to $(n-1)$ multiplied by $x$
\end{itemize}


% ================================================== %
% == List operations == %
% ================================================== %

\section{List operations}

You will do more operations on list than you have ever done before so
you might as well get used to them. These are operations that you should
know by heart.

\begin{minted}{elixir}
> [1 | []]

> [1 | [2 | []]]

> [1, 2]

> [1, 2] = [1 | [2 | []]]

> [x, y, z] = [1, 2, 3]

> [head | tail] = [1, 2, 3]

> [_, {x, y} | _] = [{:a, 1}, {:b, 2}, {:c, 3}, {:d, 4}]

> [z] = [1, 2, 3]
\end{minted}

In the above examples, what is the value of the variables after the
pattern matching expressions?

subsection{Simple functions on list}

These are some simple functions that you should implement. They will
all use recursion so first try to formulate in words what the
definition should look like, then implement it.

\begin{itemize}
\item {\tt nth(n, l)}: return the {\em n'th} element of the list {\em l}
\item {\tt len(l)}: return the number of elements in the list  {\em l}
\item {\tt sum(l)}: return the sum of all elements in the list  {\em l}, assume
  that all elements are integers
\end{itemize}

These functions take some more thinking. You should return a list as a
result of evaluating the function.

\begin{itemize}
\item {\tt duplicate(l)}: return a list where all elements are duplicated
\item {\tt add(x, l)}: add the element {\em x} to the list {\em l} if it is not in the list
\item {\tt remove(x, l)}: remove all occurrences of x in l 
\item {\tt unique(l)}: return a list of unique elements in the list {\em l},
  that is {\tt [:a, :b, :d]} are the unique elements in the list {\tt [:a, :b, :a, :d, :a, :b, :b, :a]}
\item {\tt pack(l)}: return a list containing lists of equal elements, {\tt
  [:a, :a, :b, :c, :b, :a, :c]} should return {\tt [[:a, :a, :a], [b], [:c, :c]]}
\item {\tt reverse(l)}: return a list where the order of elements is reversed
\end{itemize}


\subsection{Sorting}

There are several ways to sort a list and you should know them all. We
will start with the most basic algorithm and then try some other (more or less good).


\subsection{Insertion sort}

In {\em insertion sort}, you sort a list of elements by taking them
one at a time and {\em insert} them into an already sorted list. The
already sorted list will of course be empty when we start but will
when we are done contain all the elements.

Start by defining a function {\tt insert(element, list)}, that inserts
the element at the right place in the list. Think of the two mayor
cases, what to do if the list is empty and what to do if the list
contains at least one element. Assume that the elements are integers
and can be compared using the regular $<$ operator.

Once you have {\tt insert/2} working, implement the sorting function
{\tt isort(list, sorted)}; again what should you do if the list is
empty, what should you do if it contains at least one element?

Now all you have to do is provide a function {\tt isort(list)}, that
calls the function {\tt insert/2} using the right arguments.

\begin{minted}{elixir}
def isort(l) do 
  isort(l, ...)
end

def isort(x, l) do
  case ... do
    [] -> 
      ...
    [h | t] when h < x ->
      ...
    [h | t] ->
      ...
  end
end
\end{minted}

Try also to rewrite the {\tt isort} function using the clause syntax;
same-same but different.


\subsection{Merge sort}

In {\em merge sort}, you divide the list into two (as equal as
possible) list. Then you merge sort each of these lists to obtain two
sorted sub-lists. These sorted sub-lists are then {\em merged} into
one final sorted list. 

The two lists are merged by picking the smallest of the elements from
each of the lists. Since each list is sorted, one need only to look at
the first element of each list to determine which element is the
smallest.

The skeleton code below will give you an idea of what the solution
will look like. Here we do use the clause syntax when defining {\tt
  merge}, you can try to define it using {\em case expressions} but it
becomes a bit messy.

\pagebreak

\begin{minted}{elixir}
def msort(l) do
  case ...  do
    ... -> 
      ...
    ... ->
      {.., ...} = msplit(l, [], [])
      merge(msort(...), msort(...))
  end
end

def merge(..., ...) do ... end
def merge(..., ...) do ... end
def merge(..., ...) do
  if ...
    merge(.., ...)
  else 
    merge(.., ...)
  end
end

def msplit(..., ..., ...) do
  case ... do
    ... -> 
      {..., ...}
    ... ->
      msplit(..., ..., ...)
    end
end
\end{minted}


\subsection{Quicksort}

The {\em quicksort} algorithm sounds even quicker than merge sort but
this is not true. The idea is similar but now we ``do our sorting on
the way down''. First split the list into two parts, one containing low
elements and one containing high elements. Then sort the two lists and
when you're done append the results. 

Splitting the lists is done using the first element in the list as a
{\em pivot element}, all smaller or equal than this is added in one
list and all larger in one list. When you're appending the final
result, remember to put the pivot element in the middle.

\begin{minted}{elixir}
def qsort(...) do ... end
def qsort([p | l]) do 
  {..., ...} = qsplit(p, l, [], [])
  small = ...
  large = ...
  append(small, [p | large])
end


def qsplit(_, [], small, large) do ... end
def qsplit(p, [h | t], small, large) do
  if ...  do
    ...
  else
    ...
  end
end

def append(..., ...) do
  case ... do
    [] -> ...
    [h | t] -> ...
  end
end
\end{minted}


% ================================================== %
% == Reverse == %
% ================================================== %

\section{Reverse}

One interesting problem to look at is how to reverse a list. The {\em
  naive} way to do it is quite straight forward. We do it recursively
by removing the first element of the list, reversing the rest and then
appending the reversed list to a list containing only the first element.

\begin{minted}{elixir}
def nreverse([]) do [] end

def nreverse([h | t]) do
  r = nreverse(t)
  append(r, [h])
end
\end{minted}

A smarter way to do it, is to use an {\em accumulator} and add the
first element to this accumulator. When we have added all elements in
the lists the accumulated list is the reversed list.

\begin{minted}{elixir}
def reverse(l) do
  reverse(l, [])
end

def reverse([], r) do r end
def reverse([h | t], r) do
  reverse(t, [h | r])
end
\end{minted}

Ok, so what is so smart by doing this? This is your assignment, you
should do some performance analysis of these two functions and
describe what is happening. To have some data lead you in the right
direction and to back up your findings you should start by doing some
performance measurements.

We have here used some library functions and higher order programming
that you might not have seen so far but don't worry, you will get use
to it.

\begin{minted}{elixir}
def bench() do
  ls = [16, 32, 64, 128, 256, 512]
  n = 100
  # bench is a closure: a function with an environment.
  bench = fn(l) ->
    seq = Enum.to_list(1..l)
    tn = time(n, fn -> nreverse(seq) end)
    tr = time(n, fn -> reverse(seq) end)
    :io.format("length: ~10w  nrev: ~8w µs    rev: ~8w us~n", [l, tn, tr])
  end

  # We use the library function Enum.each that will call
  # bench(l) for each element l in ls
  Enum.each(ls, bench)
end

# Time the execution time of the a function.
def time(n, fun) do
  start = System.monotonic_time(:milliseconds)
  loop(n, fun)
  stop = System.monotonic_time(:milliseconds)
  stop - start
end

# Apply the function n times.
def loop(n, fun) do
  if n == 0 do
    :ok
  else
    fun.()
    loop(n - 1, fun)
  end
end
\end{minted}


% ================================================== %
% == More challenges == %
% ================================================== %

\section{More challenges}

You should now be up and running to take on some new challenges. When
you try these challenges, first try to express your algorithm using
recursion. Think about the simplest case and have this as your base
case. Then formulate a rule that will take you from a more complex form
one step closer to the base case.

\subsection{Integer to binary}

Implement a function that takes an integer and return its binary
representation coded as a list of ones and zeroes. The binary form of
$13$ is for example {\tt [1, 1, 0, 1]}. Converting $0$ should be trivial
so the base case should be simple. In the recursive case we can
calculate the binary representation of {\tt div(n,2)} and the append
it to either a $0$ or $1$ depending on if the number is even or odd.

\begin{minted}{elixir}
def to_binary(0) do ... end

def to_binary(n) do
  append(..., ...)
end
\end{minted}

This could be written in a better way by using an accumulator. The
accumulator will hold the binary sequence that we have determined so
far. We start with a empty list, {\tt [ ]} and add binary digits as we go.

\begin{minted}{elixir}
def to_better(n) do to_better(n, []) end

def to_better(0, b) do b end

def to_better(n, b) do
  to_better(div(n, 2), [rem(n, 2) | b])
end
\end{minted}

Why is this better than the previous one? Can you notice the difference?  Try the following:

\begin{minted}{elixir}
> Test.time(1000, fn -> Test.to_binary(123489879809809809887) end)
\end{minted}

Now try the better version - any difference? Why?

\subsection{Binary to integer}

Now try the revers and implement a function that takes a binary
representation of a number and returns an integer. To solve this it is
a lot easier to use an accumulator that holds the number of what we
have seen so far.

\begin{minted}{elixir}
def to_integer(x) do to_integer(x, ...) end

def to_integer([], n) do ... end

def to_integer([x | r], n) do
  to_integer(..., ...)
end
\end{minted}


% ================================================== %
% == Fibonacci == %
% ================================================== %

\section{Fibonacci}

The Fibonacci sequence is the sequence $0,1,1,2,3,5,8,13,21,
\ldots$. The two first numbers are $0$ and $1$ and the following
numbers are calculated by adding the two previous number. To calculate
the Fibonacci value for $42$, all you have to do is find the Fibonacci
number for $40$ and $41$ and then add them together.

Write simple Fibonacci function {\tt fib/1} and do some performance
measurements. Adapt the benchmark program above and run some tests.

\begin{minted}{elixir}
def bench_fib() do
  ls = [8,10,12,14,16,18,20,22,24,26,28,30,32]
  n = 10
  
  bench = fn(l) ->
    t = time(n, fn() -> fib(l) end)
    :io.format("n: ~4w  fib(n) calculated in: ~8w us~n", [l, t])
  end
  
  Enum.each(ls, bench)
end
\end{minted}

Find an arithmetic expression that almost describes the computation
time for $fib(n)$. Can you justify this arithmetic expression by
looking at the definition of the function?  How large Fibonacci number
do you think you can compute if you start now and let your machine run
until the seminar? First make a guess, don't try to do the calculation
in your head just make a wild guess, then try to estimate how long
time that would take using your arithmetic function, would you be able
to make it? 

Calculate a Fibonacci number as high as you possibly can.

\end{document}
  
  