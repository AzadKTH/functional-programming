\documentclass[a4paper,11pt]{article}

% Packages for page layout
\usepackage[utf8]{inputenc}
\usepackage{fancyhdr}
\usepackage{lastpage}

% Minted for syntax highlighting
\usepackage{minted}
\usemintedstyle{tango}

% Custom paragraph indentation
\setlength{\parindent}{0em}
\setlength{\parskip}{1em}
    
% Headers/footers styling
\pagestyle{fancy}
\fancyhf{}
\renewcommand{\headrulewidth}{0pt}

% Footer
\lfoot{ID1019}
\cfoot{KTH}
\rfoot{\thepage \hspace{1pt} / \pageref{LastPage}}


% SECTIONS
%
% * Concurrency
% * Tic-Tac-Toe


\begin{document}

% ================================================== %
% == Title  == %
% ================================================== %

\title{
    \textbf{Elixir Concurrency}\\
    \large{Programming II - Elixir Version}
}
\author{Johan Montelius}
\date{Spring Term 2018}
\maketitle
\thispagestyle{fancy}


% ================================================== %
% == Concurrency  == %
% ================================================== %

\section*{Concurrency}

Elixir was designed for concurrent programming. You will quickly learn
how to divide your program into communicating processes and thereby
give it far better structure. Try the following:

\begin{minted}{elixir}
defmodule Wait do

    def hello do
    receive do
        x -> IO.puts("aaa! surprise, a message: #{x}")
    end
    end

end
\end{minted}

The {\tt IO.puts} procedure will output the string to the stdout and
insert the {\tt x} value by means of string interpolation. Compile
and load the above module in the Elixir interactive shell {\tt iex} (the returned PID number may be different):

\begin{minted}{elixir}
iex(1)> c("wait.ex")
[Wait]

iex(2)> p = spawn(Wait, :hello, [])
#PID<0.92.0>
\end{minted}

The variable {\tt p} is now bound to the {\em process identifier} of
spawned process. The process was created and called the procedure {\tt
    hello/0} (this is how we name a function with zero arguments). It is
now suspended waiting for incoming messages. In the same Elixir {\tt iex} shell
execute the command:

\begin{minted}{elixir}
iex(3)> send p, "hello"
...
\end{minted}

Now register the process identifier under the name {\tt :foo} after having started a new process (the one above died after having received the message):

\begin{minted}{elixir}
iex(4)> p = spawn(Wait, :hello, [])
#PID<0.99.0>

iex(5)> Process.register(p, :foo)
true

iex(6)> send :foo, "hello"
...
\end{minted}


% ================================================== %
% == Tic-Tac-Toe  == %
% ================================================== %

\section{Tic-Tac-Toe}

In the example above the only thing we sent was a string but
we can send arbitrary complex data structures. The {\tt receive}
statement can have several clauses that try to match incoming
messages. Only if a match is found will a clause be used. Try this:

\begin{minted}{elixir}
defmodule Tic do

    def first do
    receive do
        {:tic, x} ->
        IO.puts(("tic: #{x}")
        second()
    end
    end
    
    defp second do
    receive do
        {:tac, x} ->
        IO.puts(("tac: #{x}")
        last()
        {:toe, x} ->
        IO.puts(("toe: #{x}")
        last()
    end
    end

    defp last do
    receive do
        x ->
        IO.puts(("end: #{x}")
    end
    end

end
\end{minted}

Then in the {\tt iex} shell execute the following commands:

\begin{minted}{elixir}
iex(1)> c("tic.ex")
[Tic]

iex(2)> p = spawn(Tic, :first, [])
#PID<0.103.0>

iex(3)> send p, {:toe, :bar}
...

iex(4)> send p, {:tac, :gurka}
...

iex(5)> send p, {:tic, :foo}
...
\end{minted}

In what order where they received by the process? Note how messages
are queued and how the process selects in what order to process them.

\end{document}
    
    