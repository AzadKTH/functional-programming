\documentclass[a4paper,11pt]{article}

% Packages for page layout
\usepackage[utf8]{inputenc}
\usepackage{fancyhdr}
\usepackage{lastpage}

% Minted for syntax highlighting
\usepackage{minted}
\usemintedstyle{tango}

% Custom paragraph indentation
\setlength{\parindent}{0em}
\setlength{\parskip}{1em}
 
% Headers/footers styling
\pagestyle{fancy}
\fancyhf{}
\renewcommand{\headrulewidth}{0pt}

% Footer
\lfoot{ID1019}
\cfoot{KTH}
\rfoot{\thepage \hspace{1pt} / \pageref{LastPage}}


% SECTIONS
%
% * Introduction
% * The 2-3 tree
% * The rules of insertion
% * The implementation


\begin{document}

% ================================================== %
% == Title  == %
% ================================================== %

\title{
    \textbf{A 2-3 Tree Insertion}\\
    \large{Programming II - Elixir Version}
}
\author{Johan Montelius}
\date{Spring Term 2018}
\maketitle
\thispagestyle{fancy}


% ================================================== %
% == Introduction  == %
% ================================================== %

\section*{Introduction}

You hopefully know about a tree structure called 2-3 trees. This is an
ordered tree where nodes either have two or three branches. The beauty
is that it is perfectly balanced and this balance is maintained with
not to much overhead. 

All operations (search, insert, delete) are $O(lg(n))$ and there are
no worst case scenarios since the insert and delete operations will
preserve the balance. We're not relying on luck as we are if we us a
simple binary tree.

In this assignment you're going to learn how to implement a quite
tricky algorithm and doing so using pattern matching. If you don't
remember how 2-3 trees work refresh your memory before you start.


% ================================================== %
% == The 2-3 tree  == %
% ================================================== %

\section{The 2-3 tree}

In this implementation we will create a 2-3 tree that keeps key-value
pairs. The key-values will only reside in the leafs of the tree but he
keys will of course also be present in the nodes to guide a search.

A tree is either: empty, a leaf, a two-node or a {\em three-node}. The
tree is balanced so all leafs are on the same level in the tree. A
simple representation of these different trees could be:

\begin{itemize}
    \item {\em the empty tree}: {\tt nil}
    \item {\em a leaf}: {\tt \{:leaf, key, value\}}
    \item {\em a two-node}: {\tt \{:two, key, left, right\}}
    \item {\em a three-node}: {\tt \{:three, k1, k2, left, middle, right\}}
\end{itemize}

All keys in the left branch of a two-node are smaller or equal to the
key of the node. In a three-node the keys of the left branch is
smaller of equal to the first key and the keys of the middle branch are
less than or equal to the second key.

We will in our implementation also work with a third type of node, a
node with four branches. The tree that we work with will never contain
a {\em four-node} but we will allow the insertion function to return
this structure.

\begin{itemize}
    \item {\em a four-node}: {\tt \{:four, k1, k2, k3, left, m1, m2, right\}}
\end{itemize}

Doing search in this tree is of course trivial but doing insert or
delete requires some more thinking.


% ================================================== %
% == The rules of insertion  == %
% ================================================== %

\section{The rules of insertion}

Remember the rules of the insert operation; we have some simple cases
in the bottom of the tree and then some tricky cases where we end up
with a four-node. Let's start with a description of the simple cases:

\begin{itemize}
    \item {\em empty tree}: To insert a key-value pair in an empty tree return a leaf containing the key and value.
    \item {\em a lead}: To insert a key-value pair in a leaf return a
    two-node containing the existing leaf and a new leaf. The smaller
    of the keys should be the key of the two-node.
\end{itemize}

So far, so good - now to the case were we have a two-node or a
three-node. We should do quite different thing depending on if the
node holds leafs or if it is an internal node that holds two-nodes or
three-nodes. Note that a two-node or three-node either holds only
leafs or only two- and three-nodes, it can not hold only one leaf node
since all leafs are on the same level in the tree.

So let's see what the rules are if we encounter a two- or three-node
that holds only leafs.

\begin{itemize}
    \item {\em two-node holding leafs}: Create a new leaf and return a
    three-node containing all three leafs. Make sure that the leafs
    are ordered and that the three-node hold the two smallest keys.
    \item {\em three-node holding leafs}: Cr-eat a new leaf and return a
    four-node containing all four leafs. Make sure that the leafs
    are ordered and that the four-node hold the three smallest keys.     
\end{itemize}

Now this was simple but we have of course possibly returned a
four-node. We will see how this will be removed when we do the
recursive step.

When we encounter an internal two-node or three-node will proceed and
insert the new key-value pair in the right branch. If we do this we
could end up with a four-node and then we have to think twice before
we return an new tree.

\begin{itemize}
    \item {\em two-node general case}: If the insert operation of the right branch returns a four-node, construct a valid three-node, where the middle key of the returned four-node is the new key in the three-node. Otherwise, return an updated two-node.
    \item {\em three-node general case}: If the insert operation of the right branch returns a four-node, construct a valid four-node, where the middle key of the returned four-node is the new key in the four-node. Otherwise, return an updated three-node.
\end{itemize}

The only thing that is left is the rule to handle the root of the
tree. Everything works fine with the rules that we have but it could
of course be that we will return a four-node and we don't want to have
four-nodes in our tree.

\begin{itemize}
    \item {\em the root} : If, by updating the root, we return a four-node
    then split the four-node into two two-nodes that are branches of a
    new two-node root.
\end{itemize}

That is it, now let's see how hard this is to implement.


% ================================================== %
% == The implementation  == %
% ================================================== %

\section{The implementation}

Let's implement a function {\tt insertf(key, value, tree)} that takes a
key, a value and a 2-3 tree and returns a 2-3 tree or possibly a
four-node containing 2-3 trees of equal depth (we will deal with the
root case last). Look at the rules and star with the base cases.

\begin{minted}{elixir}
def insertf(key, value, nil), do: {:leaf, key, value}

def insertf(k, v, {:leaf, k1, _} = l) do
  cond do
    k <= k1 ->
      {:two, k, {:leaf, k, v}, l}
    true ->
      {:two, k1, l, {:leaf, k, v}}
  end
end
\end{minted}

You might not have seen the construct {\tt =l} in the head but it is a
very convenient syntax. We do want the second argument to match the
pattern but at the same time we would like to use the data
structure. We could have written this without using this syntax but it
would be more to write and it would not be as efficient.

So the two first rules was easy, now for the rules were we are looking
at a two-node that holds only leafs. I leave this with some holes for
you to fill in, you should do more than just copy and paste. Look at
the rules, you should return a three-node that holds two keys and
three leafs.

\begin{minted}{elixir}
def insertf(k, v, {:two, k1, {:leaf, k1, _} = l1, 
                             {:leaf, k2, _} = l2}) do
  cond do
    k <= k1 ->
      ...
    k <= k2 ->
      ...
    true ->
      ...
  end
end
\end{minted}

The keys should of course be {\tt k1}, {\tt k2} and {\tt k}, but in
the right order and the corresponding leafs should follow. When you
think you have it, take it for a spin, terminate the clause with a dot,
compile and do some experiments. You should be able to handle a empty
tree, a tree of only a leaf and a two-node with two leafs. If you got
it proceed to the three-node.

\begin{minted}{elixir}
def insertf(k, v, {:three, k1, k2, {:leaf, k1, _} = l1,
                                   {:leaf, k2, _} = l2,
                                   {:leaf, k3, _} = l3}) do
  cond do
    k <= k1 ->
      ...
    k <= k2 ->
      ...
    k <= k3 ->
      ...
    true ->
      ...
  end
end
\end{minted}


What is the result of inserting a key-value pair in a three-node
holding three leafs? Look at the rule, complete the code, compile and
test. Ok - then the recursive cases.

\begin{minted}{elixir}
def insertf(k, v, {:two, k1, left, right}) do
  cond do
    k <= k1 ->
      case insertf(k, v, left) do
        {:four, q1, q2, q3, t1, t2, t3, t4} ->
          ...
        updated ->
          {:two, k1, updated, right}
      end

    true ->
      case insertf(k, v, right) do
        {:four, q1, q2, q3, t1, t2, t3, t4} ->
          ...
        updated ->
          {:two, k1, left, updated}
      end
  end
end
\end{minted}

Look at the first case, we have tried to insert the new key-value in
the left branch but ended up with a four-node. The four-node should be
divided into two two-nodes, one holding {\tt q1} and one holding {\tt
  q3}. The key {\tt q2} should be used to create a new three-node,
holding the two newly created two-nodes and the original right
node. The three-node should thus have the keys: {\tt q2} and {\tt k1} but
you need to put this together in the right order.

If you manage to solve the first puzzle then the second case is
easy. Again, compile and test - can you handle the test below?

\begin{minted}{elixir}
def test do
  insertf(14, :grk, {:two, 7, {:three, 2, 5, {:leaf, 2, :foo},
      {:leaf, 5, :bar}, {:leaf, 7, :zot}}, {:three, 13, 16,
      {:leaf, 13, :foo}, {:leaf, 16, :bar}, {:leaf, 18, :zot}}})
end
\end{minted}

The case where we have a three-node is of course very similar, it's a
lot of code but it should not be to difficult. You will probably make
mistakes but these are more mistakes of typos. It could be easier to
write the tree on a paper that clearly shows what {\tt q1}, {\tt q2}
etc are referring to.

\begin{minted}{elixir}
def insertf(k, v, {:three, k1, k2, left, middle, right}) do
  cond do
    k <= k1 ->
      case insertf(k, v, left) do
        {:four, q1, q2, q3, t1, t2, t3, t4} ->
          {:four, q2, k1, k2, ..., ..., ..., ...}
        updated ->
          {:three, k1, k2, ..., ..., ...}
      end
    k <= k2 ->
      case insertf(k, v, middle) do
        {:four, q1, q2, q3, t1, t2, t3, t4} ->
          ...
        updated ->
          {:three, k1, k2, ..., ..., ...}
      end
    true ->
      case insertf(k, v, right) do
        {:four, q1, q2, q3, t1, t2, t3, t4} ->
          ...
        updated ->
          {:three, k1, k2, ..., ..., ...}
      end
  end
end
\end{minted}

\end{document}
