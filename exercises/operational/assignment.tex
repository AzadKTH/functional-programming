\documentclass[a4paper,11pt]{article}

\usepackage[latin1]{inputenc}

\newcommand{\nnsection}[1]{
\section*{#1}
\addcontentsline{toc}{section}{#1}
}

\usepackage{syntax}


\begin{document}

\begin{center}
\vspace{20pt}
\textbf{\large Operational semantics of a small functional programming language.}\\
\vspace{10pt}
\textbf{Johan Montelius}\\
\vspace{10pt}
\today{}
\end{center}


\nnsection{Introduction}

The operational semantics of a language should answer the question
what an expression will evaluate to. The description should not leave
any room for interpretation since the description is what defines the
language.

The operational semantics should preferably be described in a way that
it also captures the time and memory complexity of an execution. It
does not have to be a detailed description of how things are actually
implemented but it should give an understanding of the execution to
allow a programmer to reason about the efficiency of a particular
program.

The operational semantics of a language can also serve as a
architecture for an abstract machine for the language or as the design
criteria for a compiler. The observable properties of a program
execution should conform to the properties one can derive from the
description of the operational semantics.

There are many ways to describe an operational semantics of a
programming language and we will use a strategy called {\em big-step
  semantics}. We will describe the semantics as a set of rewrite rules
or evaluation relation; given an expression in the language we ill
describe the rule of how to evaluate the expression to obtain an
answer.

This description of an operational semantics for our small functional
programming language will serve our purposes in that we will be able
to talk and reason about program execution. We will also be able to
use it when we implement an interpreter for the language.


\section{The language}

Our language is a very small functional programming language that
should behave similar to Elixir. We will in the beginning only have a
handful of data types and program constructions but will the extend
the language to look more like a proper language. To start with our
language will only consist of the constructs to express a sequence of
pattern matching expressions followed by a single expression.

\begin{figure}[ht]
\center
{\tt x = :foo, y = :nil, \{z,_\} = \{:bar,:grk\}, \{x,\{z,y\}\}.}
\caption{a simple sequence}
\label{fig:seq1}
\end{figure}

The expressions that we allow are simple {\em term expressions}
i.e. no function calls, {\em case} or {\em if expressions}. We define
this language using a BNF grammar.

We assume that we have two lexical constructs, {\em atoms} and {\em
  variables} that consist of all words written in with an initial
lowercase, atoms, and initial uppercase letters, variables.  We also
have one compound expression called a {\em cons}. This is a simple
binary construct that holds two expressions.


\begin{grammar}
<expression> ::= <atom> \alt <variable> \alt <cons> 

<cons> ::= '\{' <expression> ',' <expression> '\}'
\end{grammar}



If this was the whole language we would only be able to write
expressions like {\tt {:foo, {x, :zot}}}. This would not be that
interesting so we add the the description of a {\em sequence}.

\begin{grammar}
<sequence> ::= <expression> \alt <match> ';' <sequence>

<match> ::= <pattern> '=' <expression> 
\end{grammar}


A {\em pattern matching expression} consists of a {\em pattern} and an
expressions, and the pattern will look almost exactly as an
expression. The only difference is that we allow {\tt '_'} which will
be called {\em don't care}.

\begin{grammar}
<pattern> ::= <atom>
\alt <variable>
\alt '_'  
\alt <cons-pattern> 

<cons-pattern> ::= '\{' <pattern> ',' <pattern> '\}'
\end{grammar}

An expression is just a syntactical construct, the grammar presented
does not give any meaning to the expressions that we can write bit it
defines which sequences of characters are legal sequences, patterns
and expressions.

\section{The domain}

The {\em domain} is the set of data structures that will be the result
of our computation. It is the set of all {\em Atoms} and {\em
  compound} data structures that we can generate from elements in the
domain.

$${\rm Atoms} =  \{a,b,c, ...\}$$

$${\rm Structures} = {\rm Atoms} \cup \{c(d_1, d_2) | d_i \in {\rm Structures}\}$$

There is of course a equivalence relation between our atoms in the
language and the atoms in the domain. You might wonder if they are not
the same but there is a difference between expressions in our language
and elements in our domain. Think about the written number ``12'' or
in Roman ``XII'' and the number twelve. We will have expression that
look like {\tt \{:foo, \{:bar, :zot\}\}} and data structures that we
write {\em c(foo, c(bar, zot))}. If everything works out fine, the
{\em evaluation} of an expression will return the corresponding data
structure.

\section{An environment}

In the description of the evaluation we will need an {\em
  environment}. This is a mapping from variables in expressions to
elements in the domain. We will start with an empty environment and
gradually add more information as we evaluate a sequence of pattern
matching expressions. 

An environment is represented as a set of bindings $v/s$. Here $v$ is
a variable (and we will typically use $v$ or $x$, $y$ etc when we talk
about variables) and $t$ is a data structure. An environment that
binds X to a and Y to b would then be written:

$$\{x/a, y/b\}$$


\section{The evaluation function}

So we're now ready to describe the operational semantics of our
programming language and we do this by describing how an expressions is
{\em evaluated}. We will start with the simplest expressions and then
work our way up to more complex expressions.

The evaluation of an expression $e$ is written $E\sigma(e)$, meaning
that we evaluate the expression in the context of the environment $\sigma$.

When we describe our rules for evaluation we will use
 a notation that is:

$$\frac{{\rm prerequisite}}{E\sigma({\rm expression}) \rightarrow {\rm result}}$$

In this description $E$ is the evaluation function, and we will also
have rules for other functions and $\sigma$ an environment. The result
of applying the function is a data structure.

In order to apply the rule the prerequisite must be met and this will
in the end guide us in how we can implement a recursive evaluator. The
so called {\em big-step} operational semantics is often used since it
is easily turned into an implementation of the language.

\subsection{atoms}

The simplest rule is the one that describes how we evaluate an
expression consisting of a simple atom.

$$\frac{a \equiv s}{E\sigma(a) \rightarrow s}$$  

This mean that if we have an atom, for example {\tt :foo} then this is
evaluated to the corresponding data structure {\em foo}. The
environment $\sigma$, is in this case not relevant. 

A variable is of course different since we then need to consult the
environment for a binding of the variable.

   $$\frac{v/s \in \sigma}{E\sigma(v) \rightarrow s}$$

The rule for a compound expression {\tt \{:foo, :bar\}} is straight
forward Given that we can evaluate its two components we will of
course be able to evaluate the compound expression.

$$\frac{ E\sigma(e_1) \rightarrow s_1 \qquad   E\sigma(e_2) \rightarrow s_2}{E\sigma(\lbrace e_1 , e_2\rbrace) \rightarrow c(s_1, s_2)}$$

\subsection{a pattern matching expression}

Slightly more complex is how to evaluate a pattern matching
expression. What we need to do is to first evaluate the right hand
side and then try to match the pattern of the left hand side to the
data structure that we obtain. The result of a pattern matching is
either an extended environment or a {\em failure}. This failure is
important since we will later use it in our case statement.

We will start with one rule that might seam to be a bit redundant but
it will save us some problems when we have to write up rules for
failure. A pattern matching given a failed environment will result in
a failure.

$$\frac{}{P\;{\rm fail}\;(e, s) \rightarrow {\rm fail}}$$ 

We now proceed with the rules how to do pattern matching. The first
two rules are simple; an atom will of course match its corresponding
data structure and the don't care symbol will match anything.

$$\frac{a \equiv s}{P\sigma(a, s) \rightarrow \sigma}$$ 

$$\frac{}{P\sigma(\_,s) \rightarrow \sigma}$$

If we try to match an atom to a data structure that is not the
corresponding data structure then we fail.

$$\frac{a \not\equiv s}{P\sigma(a, s) \rightarrow {\rm fail}}$$ 

If we have a unbound variable as a pattern then the variable is bound to a
structure in the environment. 

$$\frac{v/t \not\in \sigma}{P\sigma(v, s) \rightarrow \lbrace v/s \rbrace \cup \sigma \rbrace) }$$ 

If the variable is bound to the equivalent structure we proceed
without extending the environment but if it is bound to something else
we fail.

$$\frac{v/s \in \sigma}{P\sigma(v, s) \rightarrow \lbrace \sigma \rbrace) }$$

$$\frac{v/t \in \sigma \wedge t \not\equiv s}{P\sigma(v, s) \rightarrow \lbrace {\rm fail} \rbrace) }$$ 

Matching a cons expression is quite simple but note what the rules
says about the environment. We need to do the pattern matching of the
expression $e_1$ and the data structure $s_1$ in $\sigma$ but the o
matching of $e_2$ and $s_2$ in $\sigma'$. We thus gain information in
the first matching that must be consisting with the second matching.

$$\frac{P\sigma(p_1, s_1) \rightarrow \sigma' \wedge P\sigma'(p_2, s_2) \rightarrow \theta}{P\sigma(\lbrace p_1, p_2 \rbrace  , \lbrace s_1, s_2 \rbrace) \rightarrow \theta}$$

As an exercise you should do the pattern matching of the expression
{\tt \{x, \{x, :c\}\}} and the data structure {\em c(a, c(b, c))}. Note
how we first add {\em x/a} as a binding and then {\em shadow} this binding when
we match {\tt \{x, :c\}} and {\em c(b,c)}.

The remaining alternative, the case where we have a cons expression and
we try to match this to an data structure that is not a compound data
structure, will of course lead to a failure.

$$\frac{s \not = c(s_1, s_2)}{P\sigma(\lbrace p_1, p_2 \rbrace, s) \rightarrow {\rm fail}}$$

That it is as far as pattern matching goes. Try to do some matching by
hand and explain which rules you apply. Try these:

\begin{itemize}
  
\item $P{}(\lbrace :b, :a \rbrace, \lbrace a, b \rbrace)$  
\item $P{}(\lbrace x, :b \rbrace, \lbrace a, b \rbrace)$
\item $P{}(\lbrace x, x \rbrace, \lbrace a, a \rbrace)$
\item $P{}(\lbrace x, x \rbrace, \lbrace a, b \rbrace)$
\end{itemize}


\subsection{a sequence}

So now we're ready to evaluate a sequence; a sequence that always
consist of zero or more pattern matching expressions followed by a
single expression. The pattern matching expressions will, if they
succeed, add more bindings to the environment as we proceed and the
final expression is then evaluated given this environment.

In order to describe this we introduce a new rule, a rule that
describes how a new scope is created.

$$\frac{\sigma' = \sigma \setminus \lbrace v/t \quad | \quad v/t \in \sigma \quad \wedge \quad  v \quad {\rm in} \quad p\rbrace}{S(\sigma, p) \rightarrow \sigma'}$$

The new environment will not have any bindings for the variables that
occur in the pattern. If we have a variable {\tt x} in $\sigma$ it
will simply be {\em shadowed} by the pattern matching expression.
This is quite differently from how things are handled in 
Erlang.

The rule that describes the evaluation of a sequence is now quite
straight forward. We first evaluate the expression, $e$, of the
pattern matching expression, then evaluate the matching giving us an
updated environment, $\theta$, that is used to evaluate the remaining
sequence. 

$$\frac{   
  E\sigma(e) \rightarrow t
  \qquad \sigma' = S(\sigma, p)
  \qquad P\sigma'(p, t) \rightarrow \theta
  \qquad E\theta({\rm sequence}) \rightarrow s
}{E\sigma(p = e, {\rm sequence}) \rightarrow s}$$ 

A sequence consist of one or more patter matching expressions
followed by an expression, so the rule will terminate once we reach
the final expression.

\subsection{that's it}

That is it, we now have all the rules to evaluate a sequence on the
form shown in fig:\ref{fig:seq1}. Make sure that you understand
the rules and how they are applied by evaluating the sequence by
hand. If you get it right the result will be {\em c(foo, c(bar,
  nil))}. When you get it right you're ready to continue.


\section{Adding a case expression}

The expressions that we have seen so far are rather boring. In order to
write a program that is at lest marginally interesting we need a
construct that evaluates to different data structures depending on the
state of the execution. We could have introduced a {\em if-then-else}
expression but we choose to introduce a so called {\em case expression}.

We first need to extend the grammar so that we have a syntax to express
our new construct. We choose a syntax that is similar to the case
expression in Erlang.

\begin{grammar}
     <expression> ::=  <case expression> | ...  

     <case expression> ::= 'case' <expression> 'do' <clauses>  'end' 

     <clauses> ::=   <clause> | <clause> ';' <clauses>

     <clause> ::=  <pattern> '-\textgreater' <sequence>
\end{grammar}

We then extend the rules for evaluation an expression, hopefully
capturing our intended meaning. We will use a new set of rules, $C$,
that will describe how the right clause is selected. 


$$\frac{E\sigma(e) \rightarrow t \qquad C\sigma(t, {\rm clauses}) \rightarrow s }{E\sigma({\tt case}\ e \ {\tt do}\ {\rm clauses} \ {\tt end}) \rightarrow s}$$

The right clause is selected by trying to match the pattern of the
first clause with the data structure obtained by evaluating the
expression. If the pattern matching succeeds we will continue to
evaluate the sequence of the clause, in the extended environment. Note
that we also here create new scopes for the variables in the pattern.

$$\frac{
  \sigma' = S(\sigma, p) \qquad
  P\sigma'(p, s) \rightarrow \theta \qquad
  \theta \not = {\rm fail} \qquad
  E\theta({\rm sequence}) \rightarrow s}{
C\sigma(s, p \;{\rm -\textgreater}\;    {\rm sequence} ; {\rm clauses}) \rightarrow s}$$

This rule could of course also be used even if {\em clauses} is empty
i.e. we match the last or only clause in a sequence.

If the pattern matching fails we will simply try the next clause in
the sequence of clauses.

$$\frac{
  \sigma' = S(\sigma, p) \qquad
  P\sigma'(p, s) \rightarrow {\rm fail} \qquad
  C\sigma(s, {\rm clauses}) \rightarrow s}{
C\sigma(s, p \;{\rm -\textgreater}\;  {\rm sequence} ; {\rm clauses}) \rightarrow s}$$

If no pattern matches the evaluation will simply terminate
unsuccessfully. In real life this would mean that we receive a {\em
  CaseClauseError} or similarly.

We now have everything we need to handle case expressions in our
language, this is starting to look like something.

\section{Adding lambda expressions}

The real task is when we want to add lambda expressions, or unnamed
functions. To do this wee need to do several things. We need to extend
the syntax to represent a function and to apply a function to a
sequence of arguments. We also need to add a new data structure to
represent a function and, extend the rules of evaluation to give
everything a meaning.


\subsection{free variables}

We want to know which variables in the function expression that are
{\em free}. To see the problem let's look at the function expression
that has a match expression in the sequence.

\begin{verbatim}
fn (x) -> y = 5; x + y + z end
\end{verbatim}

Which variables are {\em free} in this expression? The variable {\tt
  x} is not free since it is in the {\em scope} of the function
parameter. Nor is the local variable {\tt y} free since it is in the
scope of the pattern matching expression. If you would
translate this to lambda calculus the expression would look like
follows:

$$ \lambda x \rightarrow {\rm let} \quad y = 5 \quad {\rm in} \quad x + y + z$$

The variable {\tt z} is however free and in order to make use of this
function expression one would have to do it in an environment where
{\tt z} has a value. A function and the needed environment,
i.e. values for all free variables, is called a {\em closure}. We need
to introduce new constructs in our language to create closures and
apply them to arguments.


\subsection{function expression and application}

So we introduce two new constructs in our language, one to express a
function and one to apply a function to a sequence of
arguments. Different from Elixir we don't allow patterns in function
parameters; this is only to make the rules of the evaluation easier to
describe.

A function consist of the keyword {\tt fn} followed by a, possibly
empty, sequence of parameters (all unique variables). After the arrow
we have a regular sequence as we have defined before and everything is
finished by the keyword {\tt end}.

\begin{grammar}
<function> ::= 'fn' '(' <parameters> ')' '-\textgreater' <sequence> 'end'

<parameters> ::= ' ' | <variables> 

<variables> ::= <variable> |  <variable> ',' <variables>
\end{grammar}

A function application is simply any expression (that hopefully will
be evaluated to a {\em closure} and a sequence of arguments enclosed
in parentheses. We here follow the Elixir syntax that requires a
'.' between the name less function and the sequence of arguments.
The arguments can of course be arbitrary expressions.

\begin{grammar}
<expression> ::=  <expression>'.(' <arguments> ') | \ldots 

<arguments> ::= '  ' | <expressions> 

<expressions> ::= <expression> | <expression> ',' <expressions>
\end{grammar}

\subsection{closures}

The next thing we need to do is to extend our set of data
structures. When we evaluate a function expression we will construct
the closure. The closure is a triplet: the parameters of the function,
the sequence to evaluate and an environment.

$${\rm Closures} = \{\langle p:s:e \rangle \quad | \quad p \in {\rm Par} \wedge s \in {\rm Seq} \wedge e \in {\rm Env}\}$$

$${\rm Structures} = {\rm Closures} \cup ..$$

We have not formally defined what the set of paramters, sequences nor
environmens are but the informal description will work for our
purposes.

The environment of a closure is constructed by taking the bindings of
all free variables in the sequence, not including the variables that
are bound by the parameters.

$$\frac{ \theta = \{ v/s \mid  v/s \in \sigma \wedge v {\rm\ free\  in\ sequence}\}}{
E\sigma({\rm fn}({\rm parameters})\; {\rm -\textgreater}\; {\rm sequence}\; end )\rightarrow \quad \langle{\rm parameters}:{\rm sequence}:\theta\rangle}$$

\subsection{applying a closure}

So now to the interesting part where we apply a closure to a sequence
of arguments. We first need to evaluate the arguments and then add a
set of bindings to the environment of the closure. We then evaluate
the sequence in the updated environment.

$$\frac{E\sigma(e) \rightarrow \langle v_1, \ldots:{\rm seq}:\theta \rangle \qquad E\sigma(e_i) \rightarrow s_i \qquad E\{v_1/s_1, \ldots\}\cup\theta({\rm seq}) \rightarrow s}{E\sigma(e.(e_1, \ldots)) \rightarrow s}$$ 

Note that the closure that we get, $\langle v_1, \ldots:{\rm seq}:\theta \rangle$
consist of a sequence of variables $v_1, \ldots$, that are all
distinct. This is why we can simply add the bindings $v_i/s_i$ to
$\theta$, there will not be any duplicate bindings.

Looks complicated but it's quite straight forward. Go through the
evaluation of sequence \ref{fig:seq2}. If everything works out fine
the result should be {\em c(foo, bar)}.

\begin{figure}[ht]
\center
{\tt x = :foo; f = fn (y) -> {x,y} end;  f.(:bar)}
\caption{a sequence with a function}
\label{fig:seq2}
\end{figure}

\section{An interpreter}

The big-step operational semantics that we have used in describing the
language is very useful when one wants to implement an interpreter
for the language. If we can only come up with a scheme to represent
expressions, data structures and environments then the rules will give
us a recursive interpreter with very little effort.



\end{document}
