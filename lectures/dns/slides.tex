\input{../include/preamble}
 
 
\title[ID1019 DNS]{A DNS Resolver}
 

\author{Johan Montelius}
\institute{KTH}
\date{\semester}

\begin{document}

\begin{frame}
\titlepage
\end{frame}

\begin{frame}{Domain Name System}

\end{frame}

\begin{frame}[fragile]{RFC 1035}

\begin{verbatim}
             Local Host                        |  Foreign
                                               |
+---------+               +----------+         |  +--------+
|         | user queries  |          |queries  |  |        |
|  User   |-------------->|          |---------|->|Foreign |
| Program |               | Resolver |         |  |  Name  |
|         |<--------------|          |<--------|--| Server |
|         | user responses|          |responses|  |        |
+---------+               +----------+         |  +--------+
                            |     A            |
            cache additions |     | references |
                            V     |            |
                          +----------+         |
                          |  cache   |         |
                          +----------+         |
\end{verbatim}

\end{frame}


\begin{frame}{the resolver}

\begin{itemize}
\item client: sends request to resolver
\item resolver: receives requests, queries servers/resolvers and caches responses 
\item server: responsible for sub-domain
\end{itemize}

\vspace{10pt}\pause  

{\em The first resolver is most probably running on your laptop.}
\end{frame}


\begin{frame}[fragile]{let's build a DNS resolver}
 
\pause

\begin{verbatim}
defmodule  DNS do
\end{verbatim}

\begin{verbatim}
def start(port, server) ->
    spawn(fn() -> init(port, server) end)
end
\end{verbatim}

\vspace{10pt}\pause
{\em The server is the DNS server to which queries are routed.}

\pause

\begin{verbatim}
def init(port, server) do
   case gen_udp:open(port, [{:active, true}, :binary]) do
      {:ok, socket} ->
           dns(socket, server);
      {:error, _error} ->
           IO.puts("dns: error opening server socket")
           :ok
   end
end
\end{verbatim}


\end{frame}

\begin{frame}[fragile]{the server loop}

\begin{verbatim}
def dns(socket, server) do
    receive do
        {:udp, ^socket, ip, port, packet} ->
            Handler.handle(packet, ip, port, socket, server)
            dns(socket, server)
        :stop ->
            io:format("by by~n", [])
            :ok
        strange ->
            io:format("strange message ~w~n", [strange])
            dns(socket, server)
    end
end
\end{verbatim}

\end{frame}


\begin{frame}[fragile]{handle a request}

\pause
\begin{verbatim}
defmodule Handler do

def handle(packet, ip, port, _socket, _server) do
    :io.format("received request from ~w/~w: ~w~n", [ip, port, packet])
end
\end{verbatim}

\vspace{20pt}\pause{\em Let's try.}

\end{frame}


\begin{frame}[fragile]{forward it}

\pause
Let's forward the request to the DNS server above us.

\pause

\begin{verbatim}
def handle(packet, ip, port, socket, server) do
    case forward(packet, server) do
        {:ok, reply} ->
            gen_udp:send(socket, ip, port, reply)
        {:error, error} ->
            :io.format("error in forward ~w~n", [error])
    end
end
\end{verbatim}

\end{frame}

\begin{frame}[fragile]{forward a request}

\begin{verbatim}
def forward(request, server) do
  case gen_udp:open(0, [{:active, true}, :binary]) of
    {:ok, client} ->
      gen_udp:send(client, server, ?port, request)
      result = receive do
                 {:udp, ^client, _ip, _port, reply} ->
                    {:ok, reply}
                 after ?timeout ->
                    {:error, :timeout}
                 end
      gen_udp:close(client)
      result
    error ->
      {:error, error}
    end
end
\end{verbatim}

\end{frame}


\begin{frame}{deep inside}


\vspace{10pt}\pause
{\em Deep inside the resolver we hard code that we should use UDP packets.}

\end{frame}



\begin{frame}[fragile]{abstractions}

Hide the details, let the server module decide how things are done.

\vspace{10pt}\pause
 
\begin{verbatim}
        {:udp, socket, ip, port, packet} ->
\end{verbatim}\pause
\begin{verbatim}
            reply = fn(rep) ->  reply(rep, socket, ip, port) end
\end{verbatim}\pause
\begin{verbatim}
            forward =  fn(req) -> forward(req, server) end
\end{verbatim}\pause
\begin{verbatim}
            Handler.handle(packet, reply, forward)
\end{verbatim}\pause
\begin{verbatim}
            dns(socket, server)
\end{verbatim}

\end{frame}

\begin{frame}[fragile]{an alternative}

The handler process:
\pause
\begin{verbatim}
handle(packet, reply, forward) ->    
    case forward.(packet) of
        {:ok, msg} ->
            reply(msg);
        {:error, error} ->
            io:format("error in forward ~w~n", [error])
    end
end
\end{verbatim}

\vspace{10pt}\pause
{\em Separation of knowledge - a need to know basis.}

\end{frame}


\begin{frame}{increase throughput}

How do we increase throughput?

\end{frame}


\begin{frame}[fragile]{the server loop}

\begin{verbatim}
def dns(socket, server) do
  receive do
     {:udp, ^socket, ip, port, packet} ->
        reply = fn(rep) ->  reply(rep, socket, ip, port) end
        forward =  fn(req) -> forward(req, server) end
        Handler.start(packet, reply, forward),
        dns(socket, server)
     :stop ->
        :io.format("by by~n", []),
        :ok
     error ->
        :io.format("strange message ~w~n", [error]),
        dns(socket, server)
    end
end
\end{verbatim}

\end{frame}


\begin{frame}[fragile]{handle a request}

\pause

\begin{verbatim}
defmodule Handler do
\end{verbatim}

\pause
\begin{verbatim}
def start(packet, reply, forward) do
   spawn(fn() -> handle(packet, reply, forward) end)
end
\end{verbatim}

\pause
{\em Let's try.}
\end{frame}

\begin{frame}{encapsulation}

Encapsulation of the handling of a request in a process gives us something more than throughput - what?

\end{frame}


\begin{frame}[fragile]{let's decode the message}


  \begin{bytefield}{32}
    \bitheader{0,7-8,15-16,23-24,31} \\

    \begin{rightwordgroup}{transport \\ header}
      \bitbox{16}{identifier} &  \bitbox{16}{flags}\\
      \bitbox{16}{\# query blocks} &  \bitbox{16}{\# answer blocks }\\
      \bitbox{16}{\# authority blocks} &  \bitbox{16}{\# additional blocks }
    \end{rightwordgroup}\\
    \begin{rightwordgroup}{data \\ fields}
      \wordbox[lrt]{1}{%
        \parbox{0.6\width}{\centering \vspace{10pt} query, answer, authority \\ and additional blocks}} \\
      \skippedwords \\
      \wordbox[lrb]{1}{} 
    \end{rightwordgroup}

  \end{bytefield}

\vspace{10pt}\pause
{\em Query and response messages of the same format.}

\end{frame}

 
\begin{frame}[fragile]{message flags}

\begin{itemize}
\item QR: query or reply
\item Op-code: the operation 
\item AA: authoritative answer (if the server is responsible for the domain)
\item TC: message truncated, more to follow
\item RD: recursion desired by client 
\item RA: recursion available by server
\item Resp-code: ok or error message in response
\end{itemize}

\vspace{10pt}\pause

  \begin{bytefield}[bitwidth=2em]{16}
    \bitheader{0,1,4-5,6,7,8-9,11-12,15} \\
    \bitbox{1}{QR} & \bitbox{4}{Op-code} & \bitbox{1}{AA} & \bitbox{1}{TC} & \bitbox{1}{RD} & \bitbox{1}{RA} &  \bitbox{3}{-} &   \bitbox{4}{Resp-code} 
  \end{bytefield}


\vspace{10pt}\pause
{\em This is getting complicated.}

\end{frame}


\begin{frame}[fragile]{the bit syntax}

\begin{verbatim}
def decode(<<id:16, flags:2/binary, 
         qdc:16, anc:16, 
         ncs:16, arc:16, 
         body/binary>>=raw) do
\end{verbatim}

\pause
\begin{verbatim}
    <<qr:1, op:4, aa:1, tc:1, rd:1, ra:1, _:3, resp:4>> = flags
\end{verbatim}
\pause
\begin{verbatim}
    decoded = decode_body(qdc, anc, ncs, arc, body, raw)
\end{verbatim}
\pause
\begin{verbatim}
    {id, qr, op, aa, tc, rd, ra, rcode, decoded}
end
\end{verbatim}


{\em Why passing the raw message to the decoding of the body?}


\end{frame}

\begin{frame}[fragile]{decode the body}

The body consists of a number of: query, response, authoritative (server node) and additional sections.

\vspace{10pt}\pause

The answer, authoritative and additional sections follow the same pattern,
the query is slightly different.

\begin{verbatim}
decode_body(qdc, anc, nsc, arc, body, raw) do
    {query, rest} = decode_query(qdc, body, raw)
    {answer, rest} = decode_answer(anc, rest, raw)
    {authority, rest} = decode_answer(nsc, rest, raw)
    {additional, _} = decode_answer(arc, rest, raw)
    {query, answer, authority, additional}
end
\end{verbatim}

\vspace{10pt}\pause

{\em Note the nestling of the reminder of the body.}

\end{frame}

\begin{frame}{decode a query}

A query consists of a sequence of queries (we know from the header how many).

\begin{grammar}
<query> ::= <name> <query type> <query class>

<name> ::= <empty> | <label> <name>

<empty> ::=  0 

<label> ::=  <byte representing length n> <byte sequence of length n> \\

<query type> ::= {\em 16 bits}

<query class> ::= {\em 16 bits}

\end{grammar}

{\em The length is maximum 63 i.e. the top two bits are set to zero.}

\end{frame}

\begin{frame}[fragile]{decode a query}

\begin{verbatim}
def decode_query(0, body, _) do
    {[], body}
end
def decode_query(n, body, raw) do
    {name, <<qtype:16, qclass:16, rest/binary>>} = 
          decode_name(body, raw),
    {decoded, rest} = decode_query(n-1, rest, raw),
    {[{name, qtype, qclass} | decoded], rest}.
\end{verbatim}

\end{frame}


\begin{frame}[fragile]{decode a name}

\begin{verbatim}
def decode_name(label, raw) do
    decode_name(label, [], raw)
end
\end{verbatim}
\vspace{20pt}\pause

\begin{verbatim}
def decode_name(<<0:1, 0:1, _:6, _/binary>>=label, names, raw) do
    decode_label(label, names, raw)
end
\end{verbatim}
\end{frame}


\begin{frame}[fragile]{decode a label}

\begin{verbatim}
def decode_label(<<0:8, rest/binary>>, names, _) do
    {Enum.reverse(names), rest}
end
def decode_label(<<n:8, rest/binary>>, names, raw) do
    decode_label(n, rest, [], names, raw)
end
\end{verbatim}

\vspace{20pt}\pause

\begin{verbatim}
decode_label(0, rest, sofar, names, raw) ->
    decode_name(rest, [lists:reverse(sofar)|names], raw);

decode_label(n, <<char:8, rest/binary>>, sofar, names, raw) ->
    decode_label(n-1, rest, [char|sofar], names, raw).
\end{verbatim}

\end{frame}

\begin{frame}[fragile]{query example}

  Erlang binary:
\begin{verbatim}
  <<4,12, 1, 0,
    0, 1, 0, 0,
    0, 0, 0, 0,
    3,119,119,119,3,107,116,104,2,115,101,0,
    0,1,0,1>>
\end{verbatim}

\vspace{20pt} \pause
  Decoded query:
\begin{verbatim}
  {1036,0,0,0,0,1,0,0,{[{["www","kth","se"],1,1}],[],[],[]}}
\end{verbatim}  

\end{frame}

\begin{frame}{encoding names by offset}

The names in answers may use a more compact form of encoding.  

\vspace{20pt}\pause

Assume we have encoded {\tt www.kth.se} and need to encode {\tt
  mail.kth.se} - then we can reuse the coding of {\tt kth.se}.

\vspace{20pt} \pause

\begin{grammar}
<label> ::=  <byte representing length n> <byte sequence of length n> | \\
             <two bytes representing an offset \underline{ from the start of the message}>
\end{grammar}


\vspace{20pt} \pause


{\em The length version will always have the top two bits set to {\tt 00} and the offset version will have them set to {\tt 11}.}

\end{frame}

\begin{frame}[fragile]{offset encoding}

\begin{verbatim}
def decode_name(<<0:1, 0:1, _:6, _/binary>>=label, names, raw) do
    ## regular name encoding
    decode_label(label, names, raw)
end
      
def decode_name(<<1:1, 1:1, n:14, rest/binary>>, names, raw) do
    ## offset encoding
    offset = 8*n
    <<_:offset, section/binary>> = raw
    {name, _} = decode_label(section, names, raw)
    {name, rest}
end
\end{verbatim}

\end{frame}

\begin{frame}[fragile]{decode an answer}

All answer sections have the same basic structure:

\begin{grammar}
<answer> ::= <name> <type> <class> <ttl> <length> <resource record>
\end{grammar}

\begin{itemize}
\item type 16-bits: A-type, NS-, CNAME-, MX- etc
\item class 16-bits: Internet, ...
\item TTL 32-bits: time in seconds (typical some hours)
\item length 16-bits: the length of the record in bytes
\end{itemize}

\vspace{10pt}\pause
{\em The resource record is coded depending on the type of resource.}

\end{frame}


\begin{frame}{let's try}

Same proxy that forwards the requests but now with an effort to decode the messages.

\end{frame}

\begin{frame}{things to do}

We're wasting UDP ports - a new port is used for each query, could we do better?

\vspace{20pt}\pause

We of course want to cache the replies so that we can generate a respons directly.



\end{frame}




\end{document}
